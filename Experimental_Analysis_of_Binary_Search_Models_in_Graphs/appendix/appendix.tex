\chapter{Hardness proofs}
\begin{theorem}
    The decision version of $T||V,c,w||\sum C_j$ is NP-complete even when restricted to trees with $\Delta\br{T}\leq 16$ and to trees with $\diam\br{T}\leq 8$.
    \begin{proof}
        The problem is in NP since, given a decision tree $D$, one can verify in polynomial time 
whether all the requirements are satisfied.

To show hardness, we use a black-box reduction from the edge-query, uniform-cost, 
and non-uniform-weight variant, which is NP-complete even when restricted to trees with $\Delta\br{T}\leq 16$ and to trees with $\diam\br{T}\leq 4$ \cite{Jacobs2010OnTheComplexSearchInTsAvg}. 
Let $\br{T, w, K}$ be such an instance. 
We construct a new instance $\br{T', c, w, K'}$ for the TSP as follows: 
for every $v \in V\br{T}$, we set $c\br{v} = K+1$. 
We subdivide each edge $e \in E\br{T}$ by adding a new vertex $v_e$ with 
$w\br{v_e} = 0$ and $c\br{v_e} = 1$. 
We set $K' = K + w\br{T} \cdot \br{K+1}$.

Assume that we have a decision tree $D$ of cost at most $K$ for the original instance. 
To obtain a decision tree $D'$ for the new instance, we replace each query in $D$ 
with a query to the vertex that subdivides the corresponding edge. 
Additionally, below each leaf of $D$, we attach the appropriate queries to the original vertices. 
As $D$ contains a query to every edge of $T$, each vertex is separated, so for every 
$v \in V\br{T}$, one such additional query is added. 
This results in a decision tree $D'$ of cost at most $K + w\br{T} \cdot \br{K+1}= K'$, 
as required.

Observe that in the new instance, for every vertex $v \in T$, the cost of searching for $v$ 
is at least $K+1$ since $c\br{v} = K+1$. 
Therefore, for these vertices, at least $w\br{T} \cdot \br{K+1}$ cost is required. 
This implies that each such vertex has exactly one such query in its query sequence, 
namely the query to $v$ itself. Otherwise, the cost would exceed $K'$, and we conclude 
that every such $v$ is queried only when the candidate subset consists solely of $v$. 

Conversely, assume there exists a decision tree $D'$ of cost at most $K'$ for the new instance. 
We show how to obtain a decision tree $D$ for the original instance. 
We replace each query to a vertex $v_e$ with a query to edge $e$, 
and delete all queries to vertices $v \in V\br{T}$. 
Since each $v \in V\br{T}$ was the last query in $Q_{D'}\br{T', v}$, performed 
when the candidate set consisted only of $v$, the resulting $D$ is a valid decision tree 
for the original instance. Additionally, the cost of $D$ is at most $K' - w\br{T} \cdot \br{K+1} = K$, 
as required.

Finally, note that subdividing each edge doubles the diameter of the tree while leaving the degree unchanged, 
so the claim follows.

    \end{proof}
\end{theorem}
\begin{theorem}
    The decision version of the Weighted $\alpha$-separator Problem is (weakly) NP-complete even when restricted to stars.

\begin{tcolorbox}[colback=white, title=Decision weighted $\alpha$-separator problem, fonttitle=\bfseries, breakable]
\paragraph{Input:} Graph $G=\br{V\br{G}, E\br{G}}$, the weight function $w:V\to \mathbb{N}^+$, the cost function $c:V\to \mathbb{N}^+$, a real number $\alpha$ and an integer number $K$.
\paragraph{Output:} Whether there exists a set $S\subseteq V\br{G}$ such that for every $H\in G-S$: $w\br{H}\leq w\br{G}/\alpha$ and $c\br{S}\leq K$.
\end{tcolorbox}
    \begin{proof}
        Of course the problem is in NP since given a set $S\subseteq V\br{T}$ one can in polynomial time check whether all of the requirements are fulfilled.

        To show the hardness we use the following Partition problem which is known to be weakly NP-complete.
        \begin{tcolorbox}[colback=white, title=Partition problem, fonttitle=\bfseries, breakable]
        \paragraph{Input:} A set of integers $A = \brc{a_1,...,a_n}$.
        \paragraph{Output:} Whether there exists a subset $A'\subseteq A$ such that $\sum_{a\in A'}a=\frac{1}{2}\sum_{a\in A}a$.
        \end{tcolorbox}
        Let $A = \brc{a_1,\dots,a_n}$ be an arbitrary Partition instance. To convert it to a corresponding $\alpha$-separator instance we set $K=\frac{1}{2}\sum_{a\in A}a$, $\alpha = 2$ and create the tree $T$ with following vertices: Create a vertex $r$ such that $w\br{r} = 0$ and $c\br{r}=K + 1$. Then for each $a\in A$ create a vertex $v_a$ such that $w\br{v_a}=a$ and $w\br{c_a}=a$ and attach it $r$ by an edge $rv_a$.
        First of all, observe that $r\notin S$ since it is required that $c\br{S}\leq K$. Notice that as $\sum_{v\in V\br{T}}w\br{w}=K$ if some $v\in S$ then the $c\br{S}$ increases by $a_v$ and if otherwise for the unique $H\in T-S$: $w\br{H}$ increases by $a_v$. Therefore a valid solution for the Partition problem exists iff it is possible to partition the vertices of $T$ into two distinct subsets of which the one not containing $r$ is $S$ such that $c\br{S} = w\br{V\br{T-S}}$. The claim follows.
    \end{proof}
\end{theorem}
