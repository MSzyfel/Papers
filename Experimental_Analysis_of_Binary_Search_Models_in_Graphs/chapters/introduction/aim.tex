\section{The aim of the thesis}

Hereby, we will be mostly concerned with the situation in which the input graph is tree. A motivation for this is twofold. Firstly, trees come up most often in the practical scenarios regarding the problem. Secondly, from the algorithmic perspective, the most interesting and structural results are obtained for trees. Beyond that, most of the algorithms with provable guarantees follow some simple greedy rule and the achieved approximations are far from the objective value. For example, the problem $T||V||C_{max}$ is solvable in linear time (the algorithm is non-trivial)\cite{Schaffer1989OptNodeRankOfTsInLinTime}. If we however allow arbitrary graphs ($G||V||C_{max}$) then the problem becomes NP-hard even in chordal graphs \cite{DereniowskiVxRankOfChGsAndWTs} and the best known approximation in general case is $O\br{\log^{\frac{3}{2}} n }$ which is trivially obtained via an almost blackbox use of the tree decomposition of the graph \cite{RankingsofGraphs}. We will also be mostly concerned with the vertex query variant of the problem, since it is usually, the more general variant. 

We conclude a series of experiments aimed at verifying whether the theoretical claims regarding the discussed algorithms are reflected in an experimental setup. In particular, we employ randomized techniques to generate various classes of inputs and test the performance of the implemented algorithms both in terms of running time and the quality of the solutions obtained.
