\chapter*{Abstract}

In this work, we conduct an experimental analysis of the generalized binary search problem in graphs. The analysis explores various query models including: edge, vertex, and general queries, across multiple classes of search spaces, such as: paths, trees, general graphs, and beyond. The study is structured into two main sections:

The first part focuses on the theoretical foundations of the problem. It introduces key definitions, fundamental concepts, and pseudocodes of the analyzed procedures, along with a formal analysis of their parameters. The significance of these results was evaluated based on two primary metrics: the computational complexity and theoretical bounds on the quality of the solutions obtained.

The second part provides experimental verification of the theoretical claims established in the previous chapters. It also presents a practical comparison of the algorithmic approaches developed for different problem variants. The proposed procedures were evaluated across diverse graph classes thus ensuring complete results. To guarantee thorough and unbiased coverage of the problem space, all of the test instances were generated using randomized techniques and multiple input sizes were tested.

\paragraph{Keywords and phrases} Trees, Graph Searching, Binary Search, Decision Trees, Ranking Colorings, Graph
Theory, Approximation Algorithm, Combinatorial Optimization, Experimental Analysis of Algorithms
