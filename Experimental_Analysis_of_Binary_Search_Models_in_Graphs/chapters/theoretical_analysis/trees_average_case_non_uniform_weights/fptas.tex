\subsection{FPTAS for $T|V,w|\sum C_j$}
As it turns out one may employ a different dynamic programming technqieu to obtain an FPTAS for $T|V,w|\sum C_j$. To do so, we firstly design a pseudopolynomial time procedure which then combine with a standard rounding scheme. To do so, we begin with the following bound due to \cite{Fast_app_centroid_trees} (we managed to simplify the proof a bit):
\begin{theorem}
    Let $D^*$ be the optimal decision tree for $T|V,w|\sum C_j$. Then we have:
    $$\COST_{max, D^*}\br{T}\leq \cl{\log_{3/2}w\br{T}}$$
    \begin{proof}
        For the sake of the argument we define the following operation. Let $D$ be a decision tree for some tree $T$ and $v\in V\br{T}$. We define $D_v$ to be a decision tree such that $r\br{D}=v$. Additionally, for each $H\in T-v$ we hang $D_{|H}$ below $v$ in $D$. This operation is called a \textit{lifting} of a vertex. Let $x,v\in V\br{T}$ and $H_x\in T-v$ such that $x\in H_x$ if $x\neq v$. We have:
        $$
        Q_{D^v}\br{x}=\begin{cases}
            \brc{v} \text{if } x=v\\
            \brc{v}\cap \br{Q_{D}\br{x}\cup V\br{H}} \text{otherwise }
        \end{cases}
        $$
        We will show that after each query the size of the candidate subset decreases by a factor of $2/3$. To do so, assume contrary. Let $D$ be a minimum height decision tree for which this is not the case. By doing so we can assume that $r=r\br{D}$ has a child $c$ such that $w\br{D_y}> \frac{2w\br{T}}{3}$. Let $H_r$ denote the set of vertices not in the same component of $T-r$ as $c$ and $H_c$ denote the set of vertices not in the same component of $T-c$ as $r$. We also define $H_{r, c}=V\br{T}-H_r-H_c$. By the assumption $w\br{H_c\cup H_{r, c}}>\frac{2w\br{T}}{3}$ and $w\br{H_r}<\frac{w\br{T}}{3}$. There are two cases:
        \begin{enumerate}
            \item $w\br{H_c}>\frac{w\br{T}}{3}$. In such case we augment $D$ by lifting $c$. The query sequences of vertices in $H_c$ decrease by one query, the
query sequences of vertices in $H_r$ increase by one and query sequences of vertices in $H_{r,c}$ remain unchanged. We have:
$$
\COST_{avg, D^v}\br{T}-\COST_{avg, D}\br{T}=w\br{H_r}-w\br{H_c}< 0
$$
thus, a contradiction.
\item $w\br{H_c}\leq \frac{w\br{T}}{3}$. We have that $w\br{H_{r, c}}$ and additionally $H_{r, c}\neq\emptyset$. Let $s\in P\br{r, c}$. In such case we augment $D$ by lifting $s$. The query sequences of vertices in
$H_{c}$ remain  unchanged, since these vertices gain $t$ and lose $r$ as ancestors. The query sequences of vertices in $H_{r,c}$ are
decreased by at least one query, since each loses at least one ancestor from ${c, r}$. The query sequences of
vertices in $H_r$ increase by one, since each of these vertices gains $t$ as ancestor. We have:
$$
\COST_{avg, D^v}\br{T}-\COST_{avg, D}\br{T}=w\br{H_r}-w\br{H_{r,c}}< 0
$$
again, a contradiction.
        \end{enumerate}
As after each query to size of the candidate subset shrinks by the ratio of $2/3$ the claim follows.
    \end{proof}
\end{theorem}
The above bound on the cost allows us to assume that the height of the optimal decision tree is of order of $O\br{l\log w\br{T}}$. We will exploit the fact to build a btoom-up dynamic programming algorithm which for every $v\in V\br{T}$ will enumerate all possible "shapes" of the query sequence toward $v$. As to place the query to $v$ we only need an information about which spots in this query sequences are free and which are not, at most $2^{O\br{\log w\br{T}}}=\text{poly}\br{W}$ options need to be checked. 
