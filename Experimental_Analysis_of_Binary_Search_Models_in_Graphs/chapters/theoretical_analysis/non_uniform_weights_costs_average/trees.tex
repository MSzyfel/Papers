\subsection{$T||V,c,w||\sum C_j$}\label{serachingInTs}

In this section, we present a $\br{4+\epsilon}$-approximation algorithm 
for the case where the input graph is a tree. 
To achieve this, we establish a connection between searching in trees and 
the Weighted $\alpha$-Separator Problem. 
This connection provides a lower-bounding scheme for our recursive algorithm, 
which at each level of recursion, constructs a decision tree using the 
$\alpha$-separator obtained by the following procedure:

\begin{theorem}\label{bicriteriaFPTAS}
    Let $S$ be an optimal weighted $\alpha$-separator for $\br{T,c,w,\alpha}$. For any $\delta>0$ there exists an algorithm \FSeparatorFPTAS, which returns a separator $S'$, such that:
    \begin{enumerate}
        \item $c\br{S'}\leq c\br{S}$.
        \item $w\br{H}\leq \frac{\br{1+\delta}\cdot w\br{T}}{\alpha}$ for every $H\in T-S'$.
        \item The algorithm runs in $O\br{n^3/\delta^2}$ time.
    \end{enumerate}
\begin{proof}
We devise a dynamic programming procedure similar to the one in \cite{kseparator} and combine it with a rounding trick to obtain a bi-criteria FPTAS.
Note that the authors considered only the case in which all weights are uniform. However, we generalize their algorithm to arbitrary integer weights and introduce an additional case that 
was previously lacking\footnote{Probably due to an oversight.}.

\begin{theorem}\label{separator}
    Let $T$ be a tree. 
    There exists an optimal algorithm for the Weighted $\alpha$-Separator Problem running in 
    $O\br{n\cdot \br{w\br{T}/\alpha}^2}$ time.

\begin{proof}
    Assume that the input tree is rooted at an arbitrary vertex $r\br{T}$. Let $k=\fl{w\br{T}/\alpha}$. We want to find a separator $S$ such that for every $H\in T-S$, $w\br{H}\leq k$.
    Let $C_{v}$ denote the cost of the optimal separator $S_v$ in $T_v$ with this property. 
    Define $C_{v}^{in}$ as the cost of the optimal separator for $T_v$, 
    under the condition that $v \in S_v$. 
    We immediately have:
    $$
    C_{v}^{in} = c\br{v}+\sum_{c\in \mathcal{C}_{T,v}}C_{c}.
    $$
    
    Assume that $v \notin S_v$. 
    Let $H_v \in T_v - S_v$ be the component containing $v$. 
    For every integer $0 \leq w \leq k$, let $C_v^{out}\br{w}$ be the cost of the optimal separator for $T_v$, 
    such that $v \notin S_v$ and $w\br{H_v} = w$. 
    Then:
    $$
    C_v = \min \brc{C_{v}^{in}, \min_{0 \leq w \leq k} C_v^{out}\br{w}}.
    $$
    
    For any vertex $v \in V\br{T}$ and any integer $1 \leq i \leq \deg_{T,v}^+$, 
    let $S_{v,i}$ be the optimal separator for $T_{v,i}$ and $H_{v,i} \in T_{v,i} - S_{v,i}$ be the component containing $v$.
    For any integer $0 \leq w \leq k$, let $C_{v,i}^{out}\br{w}$ be the cost of an optimal separator for $T_{v,i}$, 
    such that $v \notin S_{v,i}$ and $w\br{H_{v,i}} = w$. 
    Then
    $$
    C_{v}^{out}\br{w} = C_{v,\deg_{T,v}^+}^{out}\br{w}.
    $$
    
    For $i = 1$ we have:
    $$
    C_{v,1}^{out}\br{w} =
    \begin{cases}
        \infty, & \text{if } w < w\br{v},\\
        \min \brc{C_{c_1}^{in},C_{c_1}^{out}\br{0}}, & \text{if } w = w\br{v},\\
        C_{c_1}^{out}\br{w - w\br{v}}, & \text{if } w > w\br{v}.
    \end{cases}
    $$
    
    For $i > 1$:
    $$
    C_{v,i}^{out}\br{w} = 
    \min \brc{
        C_{v,i-1}^{out}\br{w} + C_{c_i}^{in}, 
        \min_{0 \leq j \leq w} \brc{ C_{v,i-1}^{out}\br{w-j} + C_{c_i}^{out}\br{j} }
    }.
    $$
    
    In the above, the first term of the outer minimum corresponds to the case $c_i \in S_{v,i}$, 
    so $H_{v,i} = H_{v,i-1}$. 
    The second term considers the alternative, checking all possible partitions of 
    the weight between $H_{v,i-1}$ and $H_{c_i}$.

    These relationships suffice to compute $C_{r\br{T}}$, the cost of the optimal separator $S$ for $T$. 
    Computation is performed in a bottom-up, left-to-right manner, starting from the leaves. 
    For a leaf $v$, we have $C_v^{in} = c\br{v}$ and:
    $$C_v^{out}\br{w} = \begin{cases}
        0, & \text{if } w = w\br{v}\leq k,\\
        \infty, & \text{otherwise.}
    \end{cases}$$ 

    Since each of the $C_v^{in}$ subproblems requires $O\br{\deg_{T,v}^+}$ computational steps we get that they require $O\br{n}$ running time. As there are $O\br{n\cdot k} = O\br{n \cdot w\br{T}/\alpha}$ remaining subproblems and each requires 
    at most $O\br{k}=O\br{w\br{T}/\alpha}$ computational steps, the running time is $O\br{n \cdot \br{w\br{T}/\alpha}^2}$.
\end{proof}
\end{theorem}


Note that, the running time of the above procedure depends on $w\br{T}$ which may not be polynomial. 
To alleviate this difficulty, we slightly relax the condition on the size of components 
in $T-S$ using a controlled parameter $\delta$. 
Based on this relaxation, we show how to construct a bicriteria FPTAS for the problem. Let $\delta>0$ be any fixed constant and let $\FSeparator$ be the dynamic programming procedure from Theorem \ref{separator}.
        The algorithm is as follows:
        
\begin{algorithm}
\caption{The bicriteria FPTAS for the Weighted $\alpha$-separator Problem}
\label{createDecisionTree}
\SetKwFunction{FDecisionTree}{DecisionTree}
\SetKwProg{Fn}{Procedure}{:}{}
\Fn{$\FSeparatorFPTAS\br{T, c, w, \alpha, \delta}$}{
$K\gets\frac{\delta\cdot w\br{T}}{n\cdot \alpha}$.

\ForEach{ $v\in V\br{T}$}
{$w'\br{v} \gets \fl{\frac{w\br{v}}{K}}$.}

$\alpha'\gets\frac{\alpha\cdot K\cdot w'\br{T}}{w\br
T}$.

$S'\gets\FSeparator\br{T, c, w', \alpha'}$.

\Return $S'$.
}
\end{algorithm}
        \begin{lemma}
            Let $S$ be the optimal separator for the $\br{T, c, w, \alpha}$ instance. We have that $c\br{S'}\leq c\br{S}$.
            \begin{proof}
                We prove that $S$ is a valid separator for the $\br{T, c, w', \alpha'}$ instance, so that $c\br{S'}\leq c\br{S}$.
                To simplify the analysis, we will define the auxiliary instance: For every $v\in V\br
                {T}$, let $w''\br{v} = K\cdot\fl{\frac{w\br{v}}{K}} $. Additionally, let $ \alpha'' =\frac{\alpha \cdot w''\br{T}}{w\br{T}}$.
                
                In this new instance, for $v\in V\br{T}$ we have $w''\br{v}\leq w\br{v}$, so for every $H\in T-S$, 
                $$w''\br{H}\leq w\br{H}\leq w\br{T}/\alpha= w''\br{T}/\alpha''$$
                
                where the second inequality is by the definition of the $\alpha$-separator and the equality is by the definition of $\alpha''$.
                
                We conclude that $S$ is an $\alpha''$-separator for the auxiliary instance $\br{T, c, w'', \alpha''}$. Now notice that the $\br{T, c, w', \alpha'}$ instance has all of its weights scaled by a constant value of $K$, relatively to $\br{T, c, w'', \alpha''}$ and $\alpha' = \alpha''$. As multiplying weights by a constant does not influence the validity of a solution, $S$ is an $\alpha'$-separator for $\br{T, w', c, \alpha'}$ and the claim follows.
            \end{proof}
        \end{lemma}
        \begin{lemma}
            For every $H\in T-S'$, we have that $w\br{H}\leq\frac{\br{1+\delta}\cdot w\br{T}}{\alpha}$.
        
            \begin{proof}
                By definition $ \frac{w\br{v}}{K}\leq w'\br{v}+1$ and therefore, also $w\br{v}\leq K\cdot w'\br{v}+K$. We have:
                \begin{align*}
                \sum_{v\in H}w\br{v}&\leq K\cdot\sum_{v\in H}w'\br{v}+K\cdot n\leq \frac{K\cdot w'\br{T}}{\alpha'}+K\cdot n \\
                &= \frac{w\br{T}}{\alpha} + \frac{\delta \cdot w\br{T}}{\alpha}=\frac{\br{1+\delta}\cdot w\br{T}}{\alpha}
                \end{align*}
                
                where the second inequality is due to the fact that $S'$ is a $\alpha'$-separator for $\br{T, c, w', \alpha'}$ instance and the first equality is by the definition of $\alpha'$ and $K$.
            \end{proof}
        \end{lemma}
            
        Combining the two above lemmas with the fact that $\frac{w'\br{T}}{\alpha'}=\frac{w\br{T}}{K\cdot \alpha}=n/\delta$ we have that the algorithm runs in time $O\br{n^3/\delta^2}$ as required.
\end{proof}
\end{theorem}
\subsubsection{How to search in trees}\label{HowToSearchInTs}
Below, we show how to use the $\FSeparatorFPTAS$ procedure to construct a solution 
for $T||V,c,w||\sum C_i$. 
At each level of the recursion, the algorithm greedily finds an (almost) optimal 
weighted $\alpha$-separator of $T$, denoted $S_T$, and then builds an arbitrary 
decision tree $D_T$ using the vertices in $S_T$ (which can be done in $O\br{n^2}$ time). 

Next, for each $H \in T-S_T$, the procedure is called recursively, and each resulting 
decision tree $D_H$ is attached below the appropriate query in $D_T$. The resulting decision tree is then returned by the procedure.
    \begin{theorem}
        For any $\epsilon>0$, there exists $\br{4+\epsilon}$-approximation algorithm for $T||V,c,w||\sum C_i$ running in time $O\br{n^4/\epsilon^2}$.
        \begin{proof}
            The procedure is as follows: 
            
\begin{algorithm}[H]
\caption{The $\br{4+\epsilon}$-approximation algorithm for $T||V,c,w||\sum{C_i}$.}
\label{createDecisionTree}
\SetKwFunction{FDecisionTree}{DecisionTree}
\SetKwFunction{FSeparatorFPTAS}{SeparatorFPTAS}
\SetKwProg{Fn}{Procedure}{:}{}
\Fn{\FDecisionTree{$T, c, w,  \epsilon$}}{
$S_T\gets$ \FSeparatorFPTAS$\br{T, c, w, \alpha=2, \delta = \frac{\epsilon}{4+\epsilon}}$

Build an arbitrary decision tree $D_T$ on vertices of $S_T$.

    \ForEach{$H\in T-S$}
    {
        $D_H\gets $\FDecisionTree{$H, c, w, \epsilon$}

        Hang $D_H$ below the last query consistent with $H$ in $D_T$.
    }   
    \Return $D$
    
}
\end{algorithm}
            \begin{figure}[htbp]
    \centering
    \begin{minipage}{0.48\textwidth}
        \centering
        \begin{tikzpicture}[scale=0.7]
    \draw[thick, fill=white, drop shadow]
  (0,0) 
  .. controls (1,0) and (1,-6) .. (0,-6)  % right side curve
  .. controls (-1,-6) and (-1,0) .. (0,0); % left side curve
  
  \node at (-0.75, 0)  {$S_T$};

Dots inside
\foreach \y in {-1,-2,-3,-5} {
  \fill (0,\y) circle (4pt);
}
\node at (0, -3.9) {$\vdots$};

\draw[thick, fill=white, drop shadow]
(3,0.75) 
  .. controls (4.5,0.75) and (4.5,-0.75) .. (3,-0.75)  % right side curve
  .. controls (1.5,-0.75) and (1.5,0.75) .. (3,0.75); % left side curve

\node at (3, -0.05)  {$H_1$};

\draw[thick] (0.45,-0.5) -- (1.87,0);

\draw[thick] (0.745,-2.5) -- (2.04,-0.4);

\draw[thick, fill=white, drop shadow]
(3.5,-1) 
  .. controls (4.5,-1) and (4.5,-2) .. (3.5,-2)  % right side curve
  .. controls (2.5,-2) and (2.5,-1) .. (3.5,-1); % left side curve

\node at (3.5, -1.55)  {$H_2$};

\draw[thick] (0.66,-1.6) -- (2.74,-1.5);

\draw[thick, fill=white, drop shadow]
(3.25,-2.25) 
  .. controls (4.5,-2.25) and (4.5,-3.5) .. (3.25,-3.5)  % right side curve
  .. controls (2,-3.5) and (2,-2.25) .. (3.25,-2.25); % left side curve

\node at (3.25, -2.9)  {$H_3$};

\draw[thick] (0.75,-2.6) -- (2.32,-2.8);

\draw[thick] (0.75,-3.5) -- (2.32,-3);

\draw[thick] (0.58,-5) -- (2.6,-3.35);

\node at (3.15, -3.95) {$\vdots$};

\draw[thick, fill=white, drop shadow]
(3,-4.65) 
  .. controls (4.75,-4.65) and (4.75,-6.5) .. (3,-6.5)  % right side curve
  .. controls (1.25,-6.5) and (1.25,-4.65) .. (3,-4.65); % left side curve
  
\node at (3, -5.6)  {$H_p$};

\draw[thick] (0.69,-4.25) -- (1.85,-5.1);

\draw[thick] (0.45,-5.5) -- (1.7,-5.75);

\end{tikzpicture}
    \end{minipage}
\begin{minipage}
    {0.48\textwidth}
    \centering
    \begin{tikzpicture}[scale=1]
            \draw[thick, fill=gray!30, drop shadow] (4,-4) -- (4.9,-5.8) -- (3.1,-5.8) -- cycle
                  node[right] {$D_{T}$};
                  
            \draw[thick, fill=white, drop shadow] (2.5,-6.5) -- (3.2,-7.9) -- (1.8,-7.9) -- cycle
            node[right] {$D_{H_1}$};

            \draw[thick, fill=white, drop shadow] (3.8,-6.4) -- (4.3,-7.4) -- (3.3,-7.4) -- cycle
            node[right] {$D_{H_2}$};
            
            \draw[thick, fill=white, drop shadow] (4.8,-6.3) -- (5.2,-7.1) -- (4.4,-7.1) -- cycle
            node[right] {$D_{H_3}$};
            
            \draw[thick, fill=white, drop shadow] (6.5,-6.5) -- (7.2,-8.1) -- (5.8,-8.1) -- cycle
            node[right] {$D_{H_p}$};
            
            \node at (5.75, -7) {$\dots$};
            
            \draw[thick] (3.1,-5.8) -- (2.5,-6.5);
            \draw[thick] (3.6,-5.8) -- (3.8,-6.4);
            \draw[thick] (4.2,-5.8) -- (4.8,-6.3);
            
            \draw[thick] (4.9,-5.8) -- (6.5,-6.5);
            
        \end{tikzpicture}
\end{minipage}
    \caption{The separator $S_T$ produced by the algorithm and the structure of the decision tree built using $S_T$.}
    \label{fig:placeholder}
\end{figure}
        
            Let $\mathcal{T}$ be a subtree of $T$ for which the procedure was called and let $S_{\mathcal{T}}^*=S_{\fl{w\br{\mathcal{T}}/2}}^*\cap\mathcal{T}$. By Theorem \ref{bicriteriaFPTAS}, we have that $c\br{S_{\mathcal{T}}}\leq c\br{S_{\mathcal{T}}^*}$. Using $\beta=\frac{1-\delta}{2}$ and applying Lemma \ref{splitting} we have that the contribution of the decision tree $D_{\mathcal{T}}$ is bounded by:
            \begin{align*}
                w\br{\mathcal{T}}\cdot c\br{S_{\mathcal{T}}}
                &\leq w\br{\mathcal{T}}\cdot c\br{S_{\mathcal{T}}^*}\leq \frac{2}{1-\delta}\cdot \sum_{k=\frac{1+\delta}{2}\cdot w\br{\mathcal{T}}+1}^{w\br{\mathcal{T}}}c\br{S_{\fl{k/2}}^*\cap \mathcal{T}}.
            \end{align*}
            
            To bound the cost of the whole solution we will firstly show the following lemma which is necessary to proceed:
            \begin{lemma}\label{up_trees}
            $$\sum_{\mathcal{T}}\sum_{k=\frac{1+\delta}{2}\cdot w\br{\mathcal{T}}+1}^{w\br{\mathcal{T}}}c\br{S_{\fl{k/2}}^*\cap \mathcal{T}}\leq \sum_{k=0}^{w\br{T}}c\br{S_{\fl{k/2}}^*}.$$
            \end{lemma}
            \begin{proof}
                Fix a value of $\mathcal{T}$ and $k$. Their contribution to the cost is $c\br{S_{\fl{k/2}}^*\cap \mathcal{T}}$. Consider which candidate subtrees contribute such a term. As $S_{\mathcal{T}}$ is a weighted $\frac{2}{1+\delta}$-separator, we have that $\mathcal{T}$ is the minimal candidate subtree, such that $w\br{\mathcal{T}}\geq k\geq \frac{\br{1+\delta}\cdot w\br{\mathcal{T}}}{2}+1 >w\br{H}$, for every $H\in \mathcal{T}-S_{\mathcal{T}}$. This means that if for every $H\in \mathcal{T}-S_{\mathcal{T}}$, $w\br{H}<k$, then $\mathcal{T}$ contributes such a term. Since for all $H_1, H_2\in \mathcal{T}-S_{\mathcal{T}}$ we have that $H_1\cap H_2=\emptyset$, $\br{S_{\fl{k/2}}^*\cap H_1}\cup \br{S_{\fl{k/2}}^*\cap H_2}=\emptyset$, the claim follows by summing over all values of $k$.
            \end{proof}
            
        We are now ready to bound the cost of the solution. Let $D$ be the decision tree returned by the procedure. Using the fact that by definition $\frac{4}{1-\delta}=4+\epsilon$, we have:
        \begin{align*}
            $\COST_{D}\br{T} &\leq \sum_{\mathcal{T}} w\br{\mathcal{T}}\cdot c\br{S_\mathcal{T}}
            \leq \frac{2}{1-\delta}\cdot\sum_{\mathcal{T}}\sum_{k=\frac{1+\delta}{2}\cdot w\br{\mathcal{T}}+1}^{w\br{\mathcal{T}}}c\br{S_{\fl{k/2}}^*\cap \mathcal{T}}\\
            &\leq \frac{2}{1-\delta}\cdot\sum_{k=0}^{w\br{T}}c\br{S_{\fl{k/2}}^*}\leq \frac{4}{1-\delta}\cdot\OPT\br{T} = \br{4+\epsilon}\cdot\OPT\br{T}
    \end{align*}

    where the third inequality is due to Lemma \ref{up_trees} and the last inequality is by Lemma \ref{lb_opt}.

    
    As $1/\delta=\frac{4+\epsilon}{\epsilon}=1+4/\epsilon$ and each $v\in V\br T$ belongs to the set $S_{\mathcal{T}}$ exactly once, we have that the overall running time is at most $O\br{n^4/\epsilon^2}$ as required.
        \end{proof}
    \end{theorem}

