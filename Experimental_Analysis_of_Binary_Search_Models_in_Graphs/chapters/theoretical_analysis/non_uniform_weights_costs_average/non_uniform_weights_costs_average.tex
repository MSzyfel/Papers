\newpage
\section{Non-uniform weights and costs, average case}
In this section we will be concerned with the $T||V,c,w||\sum C_i$ variant problem which is NP-hard, by an easy reduction from the $T||E,w||\sum C_i$ problem \cite{Jacobs2010OnTheComplexSearchInTsAvg}.
We introduce the following reinterpretation of the cost function, 
for each node $v \in D$, let $T_{D,v}$ be the subtree of $T$ in which 
$v$ is queried when using $D$. 
Then, the contribution of $v$ to the total cost is $w\br{T_{D,v}} \cdot c\br{v}$, 
and therefore we obtain the following simple lemma:

\begin{lemma}\label{contributionLemma}
$$
c_T\br{D} = \sum_{v \in V\br{T}} w\br{T_{D,v}} \cdot c\br{v}.
$$  
\end{lemma}
\subsection{Cuts and separators}\label{cutsAndSeparators}
To obtain a tight lower bound on the cost of our solution, 
we establish a connection between the $T||V,c,w||\sum C_i$ and the following 
vertex separator problems. 
We define the \textit{Weighted $\alpha$-Separator Problem} as follows:


\begin{tcolorbox}[colback=white, title=Weighted $\alpha$-Separator Problem, fonttitle=\bfseries, breakable]
\textbf{Input:} Tree $T$, a cost function $c:V\to \mathbb{N}$, a weight function $w:V\to \mathbb{N}$ and a real number $\alpha$.

\textbf{Output:} A set $S\subseteq V\br{T}$ called \textit{$\alpha$-separator}, such that for every $H\in T-S$, $w\br{H}\leq w\br{T}/\alpha$ and $c\br{S}$ is minimized.
\end{tcolorbox}
% % We also define the Min-Ratio Vertex Cut Problem as follows:

% % \begin{tcolorbox}[colback=white, title= Min-Ratio Vertex Cut Problem, fonttitle=\bfseries, breakable]
% % \textbf{Input:} Graph $G=\br{V\br{G}, E\br{G}}$, the cost function $c:V\to \mathbb{N}$ and the weight function $w:V\to \mathbb{N}$.

% % \textbf{Output:} A partition $\br{A,S,B}$ of $V\br{G}$ called \textit{vertex-cut}, such that there are no $u\in A$ and $v\in B$ for which $uv\in E\br{G}$, minimizing the ratio:
% % $$
% % \alpha_{c,w}\br{A,S,B}=\frac{c\br{S}}{w\br{A\cup S}\cdot w\br{B\cup S}}.
% % $$
% % \end{tcolorbox}
\subsection{Levels of $\OPT$ and basic bounds}\label{levelsOfOptAndBasicBounds}

We begin with additional notation. 
For any tree $T$ and decision tree $D$, denote by 
$\mathcal{R}_D\br{T} = \brc{V\br{T_{D,v}}\colon v \in V\br{T}}$ 
the family of all candidate subsets of $D$ in $T$. 

Let $D^*$ be an arbitrary decision tree for the $G||V,c,w||\sum C_i$ such that 
$\COST_{D^*}\br{T} = \OPT\br{T}$. 
We denote by $\mathcal{L}_{k}^*$ the subfamily of $\mathcal{R}_{D^*}\br{T}$ 
consisting of all maximal elements $H$ of $\mathcal{R}_{D^*}\br{T}$ with $w\br{H} \leq k$, 
that is, if some superset $H'$ of $H$ belongs to $\mathcal{R}_{D^*}\br{T}$, 
then $w\br{H'} > k$. 
We call such a set the $k$-th \textit{level} of $\OPT\br{T}$. 
Let $S_{k}^* = V\br{T} - \mathcal{L}_{k}^*$. 
These are the vertices belonging to the separator at level $\mathcal{L}_{k}^*$. 

Notice that $S_{k}^*$ forms a Weighted $w\br{T}/k$-separator of $T$. 
Furthermore, for any $H_1, H_2 \in \mathcal{R}_D\br{T}$, we have 
$H_1 \cup H_2 \neq \emptyset$ if and only if $H_1 \subseteq H_2$ or 
$H_2 \subseteq H_1$, so $\mathcal{R}_D\br{T}$ is laminar. 
Therefore, for any $k_1 \neq k_2$, we have 
$\mathcal{L}_{k_1}^* \cap \mathcal{L}_{k_2}^* = \emptyset$.
\begin{lemma}
                $$\OPT\br{T}=\sum_{k=0}^{w\br{T}-1}c\br{S_{k}^*}.$$
                
            \begin{proof}
                Consider any vertex $v$. For every $0\leq k<w\br{G_{D^*,v}}$, $v\notin \bigcup_{H\in \mathcal{L}_{k}^*}H$, so $v\in S_{k}^*$ and the contribution of $v$ to the cost is $w\br{G_{D^*,v}}\cdot c\br{v}$:
                $$\sum_{k=0}^{w\br{T}-1}c\br{S_{k}^*}=\sum_{v\in V\br{T}}\sum_{k=0}^{w\br{G_{D^*,v}}-1}c\br{v}=\OPT\br{T}$$
                
                where the second equality is by Lemma \ref{contributionLemma}.
            \end{proof}
            \end{lemma}
            
Using the above lemma one easily obtains the following lower bound on the cost of the optimal solution:
\begin{lemma}\label{lb_opt}
            $$
            2\cdot\OPT\br{T}= 2\cdot\sum_{k=0}^{w\br{T}-1}c\br{S_{k}^*} \geq \sum_{k=0}^{w\br{T}}c\br{S_{\fl{k/2}}^*}.
            $$
\end{lemma}

We also have the following upper bound:
\begin{lemma}\label{splitting}
    Let $\mathcal{T}$ be any subtree of $T$ and $0\leq\beta\leq 1$. Then: 
            $$
           \beta\cdot w\br{\mathcal{T}}\cdot c\br{S_{\fl{w\br{\mathcal{T}}/2}}^*\cap \mathcal{T}}
            \leq \sum_{k=\br{1-\beta}w\br{\mathcal{T}}+1}^{w\br{\mathcal{T}}}c\br{S_{\fl{k/2}}^*\cap \mathcal{T}}.
            $$
    \begin{proof}
        The inequality is due to the fact that as $k$ decreases, more vertices belong to the separator. 
    \end{proof}
\end{lemma}

\subsection{A $\br{4+\epsilon}$-approximation for $T||V,c,w||\sum C_j$}\label{serachingInTs}

In this section, we present a $\br{4+\epsilon}$-approximation algorithm 
for the case where the input graph is a tree. 
To achieve this, we establish a connection between searching in trees and 
the Weighted $\alpha$-Separator Problem. 
This connection provides a lower-bounding scheme for our recursive algorithm, 
which at each level of recursion, constructs a decision tree using the 
$\alpha$-separator obtained by the following procedure:

\begin{theorem}\label{bicriteriaFPTAS}
    Let $S$ be an optimal weighted $\alpha$-separator for $\br{T,c,w,\alpha}$. For any $\delta>0$ there exists an algorithm \FSeparatorFPTAS, which returns a separator $S'$, such that:
    \begin{enumerate}
        \item $c\br{S'}\leq c\br{S}$.
        \item $w\br{H}\leq \frac{\br{1+\delta}\cdot w\br{T}}{\alpha}$ for every $H\in T-S'$.
        \item The algorithm runs in $O\br{n^3/\delta^2}$ time.
    \end{enumerate}
\begin{proof}
We devise a dynamic programming procedure similar to the one in \cite{kseparator} and combine it with a rounding trick to obtain a bi-criteria FPTAS.
Note that the authors considered only the case in which all weights are uniform. However, we generalize their algorithm to arbitrary integer weights and introduce an additional case that 
was previously lacking\footnote{Probably due to an oversight.}.

\begin{theorem}\label{separator}
    Let $T$ be a tree. 
    There exists an optimal algorithm for the Weighted $\alpha$-Separator Problem running in 
    $O\br{n\cdot \br{w\br{T}/\alpha}^2}$ time.

\begin{proof}
    Assume that the input tree is rooted at an arbitrary vertex $r\br{T}$. Let $k=\fl{w\br{T}/\alpha}$. We want to find a separator $S$ such that for every $H\in T-S$, $w\br{H}\leq k$.
    Let $C_{v}$ denote the cost of the optimal separator $S_v$ in $T_v$ with this property. 
    Define $C_{v}^{in}$ as the cost of the optimal separator for $T_v$, 
    under the condition that $v \in S_v$. 
    We immediately have:
    $$
    C_{v}^{in} = c\br{v}+\sum_{c\in \mathcal{C}_{T,v}}C_{c}.
    $$
    
    Assume that $v \notin S_v$. 
    Let $H_v \in T_v - S_v$ be the component containing $v$. 
    For every integer $0 \leq w \leq k$, let $C_v^{out}\br{w}$ be the cost of the optimal separator for $T_v$, 
    such that $v \notin S_v$ and $w\br{H_v} = w$. 
    Then:
    $$
    C_v = \min \brc{C_{v}^{in}, \min_{0 \leq w \leq k} C_v^{out}\br{w}}.
    $$
    
    For any vertex $v \in V\br{T}$ and any integer $1 \leq i \leq \deg_{T,v}^+$, 
    let $S_{v,i}$ be the optimal separator for $T_{v,i}$ and $H_{v,i} \in T_{v,i} - S_{v,i}$ be the component containing $v$.
    For any integer $0 \leq w \leq k$, let $C_{v,i}^{out}\br{w}$ be the cost of an optimal separator for $T_{v,i}$, 
    such that $v \notin S_{v,i}$ and $w\br{H_{v,i}} = w$. 
    Then
    $$
    C_{v}^{out}\br{w} = C_{v,\deg_{T,v}^+}^{out}\br{w}.
    $$
    
    For $i = 1$ we have:
    $$
    C_{v,1}^{out}\br{w} =
    \begin{cases}
        \infty, & \text{if } w < w\br{v},\\
        \min \brc{C_{c_1}^{in},C_{c_1}^{out}\br{0}}, & \text{if } w = w\br{v},\\
        C_{c_1}^{out}\br{w - w\br{v}}, & \text{if } w > w\br{v}.
    \end{cases}
    $$
    
    For $i > 1$:
    $$
    C_{v,i}^{out}\br{w} = 
    \min \brc{
        C_{v,i-1}^{out}\br{w} + C_{c_i}^{in}, 
        \min_{0 \leq j \leq w} \brc{ C_{v,i-1}^{out}\br{w-j} + C_{c_i}^{out}\br{j} }
    }.
    $$
    
    In the above, the first term of the outer minimum corresponds to the case $c_i \in S_{v,i}$, 
    so $H_{v,i} = H_{v,i-1}$. 
    The second term considers the alternative, checking all possible partitions of 
    the weight between $H_{v,i-1}$ and $H_{c_i}$.

    These relationships suffice to compute $C_{r\br{T}}$, the cost of the optimal separator $S$ for $T$. 
    Computation is performed in a bottom-up, left-to-right manner, starting from the leaves. 
    For a leaf $v$, we have $C_v^{in} = c\br{v}$ and:
    $$C_v^{out}\br{w} = \begin{cases}
        0, & \text{if } w = w\br{v}\leq k,\\
        \infty, & \text{otherwise.}
    \end{cases}$$ 

    Since each of the $C_v^{in}$ subproblems requires $O\br{\deg_{T,v}^+}$ computational steps we get that they require $O\br{n}$ running time. As there are $O\br{n\cdot k} = O\br{n \cdot w\br{T}/\alpha}$ remaining subproblems and each requires 
    at most $O\br{k}=O\br{w\br{T}/\alpha}$ computational steps, the running time is $O\br{n \cdot \br{w\br{T}/\alpha}^2}$.
\end{proof}
\end{theorem}


Note that, the running time of the above procedure depends on $w\br{T}$ which may not be polynomial. 
To alleviate this difficulty, we slightly relax the condition on the size of components 
in $T-S$ using a controlled parameter $\delta$. 
Based on this relaxation, we show how to construct a bicriteria FPTAS for the problem. Let $\delta>0$ be any fixed constant and let $\FSeparator$ be the dynamic programming procedure from Theorem \ref{separator}.
        The algorithm is as follows:
        
\begin{algorithm}
\caption{The bicriteria FPTAS for the Weighted $\alpha$-separator Problem}
\label{createDecisionTree}
\SetKwFunction{FDecisionTree}{DecisionTree}
\SetKwProg{Fn}{Procedure}{:}{}
\Fn{$\FSeparatorFPTAS\br{T, c, w, \alpha, \delta}$}{
$K\gets\frac{\delta\cdot w\br{T}}{n\cdot \alpha}$.

\ForEach{ $v\in V\br{T}$}
{$w'\br{v} \gets \fl{\frac{w\br{v}}{K}}$.}

$\alpha'\gets\frac{\alpha\cdot K\cdot w'\br{T}}{w\br
T}$.

$S'\gets\FSeparator\br{T, c, w', \alpha'}$.

\Return $S'$.
}
\end{algorithm}
        \begin{lemma}
            Let $S$ be the optimal separator for the $\br{T, c, w, \alpha}$ instance. We have that $c\br{S'}\leq c\br{S}$.
            \begin{proof}
                We prove that $S$ is a valid separator for the $\br{T, c, w', \alpha'}$ instance, so that $c\br{S'}\leq c\br{S}$.
                To simplify the analysis, we will define the auxiliary instance: For every $v\in V\br
                {T}$, let $w''\br{v} = K\cdot\fl{\frac{w\br{v}}{K}} $. Additionally, let $ \alpha'' =\frac{\alpha \cdot w''\br{T}}{w\br{T}}$.
                
                In this new instance, for $v\in V\br{T}$ we have $w''\br{v}\leq w\br{v}$, so for every $H\in T-S$, 
                $$w''\br{H}\leq w\br{H}\leq w\br{T}/\alpha= w''\br{T}/\alpha''$$
                
                where the second inequality is by the definition of the $\alpha$-separator and the equality is by the definition of $\alpha''$.
                
                We conclude that $S$ is an $\alpha''$-separator for the auxiliary instance $\br{T, c, w'', \alpha''}$. Now notice that the $\br{T, c, w', \alpha'}$ instance has all of its weights scaled by a constant value of $K$, relatively to $\br{T, c, w'', \alpha''}$ and $\alpha' = \alpha''$. As multiplying weights by a constant does not influence the validity of a solution, $S$ is an $\alpha'$-separator for $\br{T, w', c, \alpha'}$ and the claim follows.
            \end{proof}
        \end{lemma}
        \begin{lemma}
            For every $H\in T-S'$, we have that $w\br{H}\leq\frac{\br{1+\delta}\cdot w\br{T}}{\alpha}$.
        
            \begin{proof}
                By definition $ \frac{w\br{v}}{K}\leq w'\br{v}+1$ and therefore, also $w\br{v}\leq K\cdot w'\br{v}+K$. We have:
                \begin{align*}
                \sum_{v\in H}w\br{v}&\leq K\cdot\sum_{v\in H}w'\br{v}+K\cdot n\leq \frac{K\cdot w'\br{T}}{\alpha'}+K\cdot n \\
                &= \frac{w\br{T}}{\alpha} + \frac{\delta \cdot w\br{T}}{\alpha}=\frac{\br{1+\delta}\cdot w\br{T}}{\alpha}
                \end{align*}
                
                where the second inequality is due to the fact that $S'$ is a $\alpha'$-separator for $\br{T, c, w', \alpha'}$ instance and the first equality is by the definition of $\alpha'$ and $K$.
            \end{proof}
        \end{lemma}
            
        Combining the two above lemmas with the fact that $\frac{w'\br{T}}{\alpha'}=\frac{w\br{T}}{K\cdot \alpha}=n/\delta$ we have that the algorithm runs in time $O\br{n^3/\delta^2}$ as required.
\end{proof}
\end{theorem}
\subsubsection{How to search in trees}\label{HowToSearchInTs}
Below, we show how to use the $\FSeparatorFPTAS$ procedure to construct a solution 
for $T||V,c,w||\sum C_i$. 
At each level of the recursion, the algorithm greedily finds an (almost) optimal 
weighted $\alpha$-separator of $T$, denoted $S_T$, and then builds an arbitrary 
decision tree $D_T$ using the vertices in $S_T$ (which can be done in $O\br{n^2}$ time). 

Next, for each $H \in T-S_T$, the procedure is called recursively, and each resulting 
decision tree $D_H$ is attached below the appropriate query in $D_T$. The resulting decision tree is then returned by the procedure.
    \begin{theorem}
        For any $\epsilon>0$, there exists $\br{4+\epsilon}$-approximation algorithm for $T||V,c,w||\sum C_i$ running in time $O\br{n^4/\epsilon^2}$.
        \begin{proof}
            The procedure is as follows: 
            
\begin{algorithm}
\caption{The $\br{4+\epsilon}$-approximation algorithm for $T||V,c,w||\sum C_i$}
\label{createDecisionTree}
\SetKwFunction{FDecisionTree}{DecisionTree}
\SetKwFunction{FSeparatorFPTAS}{SeparatorFPTAS}
\SetKwProg{Fn}{Procedure}{:}{}
\Fn{$\FDecisionTree\br{T, c, w,  \epsilon}$}{
$S_T\gets\FSeparatorFPTAS\br{T, c, w, \alpha=2, \delta = \frac{\epsilon}{4+\epsilon}}$.

$D_T\gets$ arbitrary partial decision tree for $T$, built from vertices of $S_T$.

    \ForEach{$H\in T-S_T$}
    {
        $D_H\gets \FDecisionTree\br{H, c, w, \epsilon}$.

        Hang $D_H$ in $D_T$ below the last query to $v\in N_T\br{H}$.
    }   
    \Return $D_T$.
    
}
\end{algorithm}
            \input{figures/average_case_structure.tex}
        
            Let $\mathcal{T}$ be a subtree of $T$ for which the procedure was called and let $S_{\mathcal{T}}^*=S_{\fl{w\br{\mathcal{T}}/2}}^*\cap\mathcal{T}$. By Theorem \ref{bicriteriaFPTAS}, we have that $c\br{S_{\mathcal{T}}}\leq c\br{S_{\mathcal{T}}^*}$. Using $\beta=\frac{1-\delta}{2}$ and applying Lemma \ref{splitting} we have that the contribution of the decision tree $D_{\mathcal{T}}$ is bounded by:
            \begin{align*}
                w\br{\mathcal{T}}\cdot c\br{S_{\mathcal{T}}}
                &\leq w\br{\mathcal{T}}\cdot c\br{S_{\mathcal{T}}^*}\leq \frac{2}{1-\delta}\cdot \sum_{k=\frac{1+\delta}{2}\cdot w\br{\mathcal{T}}+1}^{w\br{\mathcal{T}}}c\br{S_{\fl{k/2}}^*\cap \mathcal{T}}.
            \end{align*}
            
            To bound the cost of the whole solution we will firstly show the following lemma which is necessary to proceed:
            \begin{lemma}\label{up_trees}
            $$\sum_{\mathcal{T}}\sum_{k=\frac{1+\delta}{2}\cdot w\br{\mathcal{T}}+1}^{w\br{\mathcal{T}}}c\br{S_{\fl{k/2}}^*\cap \mathcal{T}}\leq \sum_{k=0}^{w\br{T}}c\br{S_{\fl{k/2}}^*}.$$
            \end{lemma}
            \begin{proof}
                Fix a value of $\mathcal{T}$ and $k$. Their contribution to the cost is $c\br{S_{\fl{k/2}}^*\cap \mathcal{T}}$. Consider which candidate subtrees contribute such a term. As $S_{\mathcal{T}}$ is a weighted $\frac{2}{1+\delta}$-separator, we have that $\mathcal{T}$ is the minimal candidate subtree, such that $w\br{\mathcal{T}}\geq k\geq \frac{\br{1+\delta}\cdot w\br{\mathcal{T}}}{2}+1 >w\br{H}$, for every $H\in \mathcal{T}-S_{\mathcal{T}}$. This means that if for every $H\in \mathcal{T}-S_{\mathcal{T}}$, $w\br{H}<k$, then $\mathcal{T}$ contributes such a term. Since for all $H_1, H_2\in \mathcal{T}-S_{\mathcal{T}}$ we have that $H_1\cap H_2=\emptyset$, $\br{S_{\fl{k/2}}^*\cap H_1}\cup \br{S_{\fl{k/2}}^*\cap H_2}=\emptyset$, the claim follows by summing over all values of $k$.
            \end{proof}
            
        We are now ready to bound the cost of the solution. Let $D$ be the decision tree returned by the procedure. Using the fact that by definition $\frac{4}{1-\delta}=4+\epsilon$, we have:
        \begin{align*}
            $\COST_{D}\br{T} &\leq \sum_{\mathcal{T}} w\br{\mathcal{T}}\cdot c\br{S_\mathcal{T}}
            \leq \frac{2}{1-\delta}\cdot\sum_{\mathcal{T}}\sum_{k=\frac{1+\delta}{2}\cdot w\br{\mathcal{T}}+1}^{w\br{\mathcal{T}}}c\br{S_{\fl{k/2}}^*\cap \mathcal{T}}\\
            &\leq \frac{2}{1-\delta}\cdot\sum_{k=0}^{w\br{T}}c\br{S_{\fl{k/2}}^*}\leq \frac{4}{1-\delta}\cdot\OPT\br{T} = \br{4+\epsilon}\cdot\OPT\br{T}
    \end{align*}

    where the third inequality is due to Lemma \ref{up_trees} and the last inequality is by Lemma \ref{lb_opt}.

    
    As $1/\delta=\frac{4+\epsilon}{\epsilon}=1+4/\epsilon$ and each $v\in V\br T$ belongs to the set $S_{\mathcal{T}}$ exactly once, we have that the overall running time is at most $O\br{n^4/\epsilon^2}$ as required.
        \end{proof}
    \end{theorem}


% \subsection{$G||V,c,w||\sum C_j$}\label{serachingInGs}
To construct decision trees for general graphs, we exploit a connection to a different problem, 
namely the Min-Ratio Vertex Cut Problem. 
Let $\alpha_{c,w}\br{G}$ denote the optimal value of such a vertex cut. 
We invoke the following result given in \cite{Improvedapproximationalgorithmsvertexseparators}:

\begin{theorem}\label{approxmrvc}
    Given a graph $G=\br{V\br{G}, E\br{G}}$, the cost function $c:V\to\mathbb{N}$ and the weight function $w:V\to \mathbb{N}$, there exists a
polynomial-time algorithm, which computes a partition $(A, S, B)$, such that:
$$
\alpha_{c,w}\br{A,S,B}=O\br{\sqrt{\log n
}}\cdot\alpha_{c,w}\br{G}.
$$
\end{theorem}

Combining the latter procedure, with the following result, yields an $O\br{\sqrt{\log n
}}$-approximation algorithm for the $G||V,c,w||\sum C_i$:
\begin{theorem}
    Let $f_n$ be the approximation ratio of any polynomial time algorithm for the Min-Ratio Vertex Cut Problem. Then, there exists an $O\br{f_n}$-approximation algorithm for the GSP, running in polynomial time.
    \begin{proof}
        Let $\FAlgorithmMinCut$ be the procedure which achieves the $f_n$-approximation ratio for the Min-Ratio Vertex Cut Problem.
        The algorithm is as follows:
        
\begin{algorithm}[H]
\caption{The $f_n$-approximation algorithm for $G||V,c,w||\sum C_i$.}
\SetKwFunction{FDecisionTree}{DecisionTree}
\SetKwProg{Fn}{Procedure}{:}{}
\Fn{$\FDecisionTree\br{G,c,w}$}{
$A_G,S_G, B_G\gets\FAlgorithmMinCut\br{G, c, w}$.

$D_G\gets$ arbitrary partial decision tree for $G$, built from vertices of $S_G$.

    \ForEach{$H\in G-S_G$}
    {
        $D_H\gets \FDecisionTree\br{H, c, w}$.

        Hang $D_H$ in $D_G$ below the last query to $v\in N_G\br{H}$.
    }   
    \Return $D_G$.
    
}
\end{algorithm}
        
        Let $\mathcal{G}$ be any subgraph of $G$, for which the procedure was called and let $S_{\mathcal{G}}^*=S_{\fl{w\br{\mathcal{G}}/2}}^*\cap \mathcal{G}$.
        \begin{lemma}\label{lambda_lemma}
            Let $\mathcal{H}=\mathcal{G}-S_{\mathcal{G}}^*$ and let $\lambda=6+2\sqrt{5}$ be the unique, positive solution of the equation $\frac{1}{4}-\frac{1}{2\sqrt{\lambda}}=\frac{1}{\lambda}$. Then, we can partition $\mathcal{H}$ into two sets, $\mathcal{A}$ and $\mathcal{B}$ such that for $A=\bigcup_{H\in\mathcal{A}}V\br{H}$ and $B=\bigcup_{H\in\mathcal{B}}V\br{H}$, we have:
            $$w\br{A\cup S_{\mathcal{G}}^*}\cdot w\br{B\cup S_{\mathcal{G}}^*}\geq w\br{\mathcal{G}}^2/\lambda.$$
            
            \begin{proof}
                There are two cases:
                \begin{enumerate}
                    \item $w\br{ S_{\mathcal{G}}^*}\geq w\br{\mathcal{G}}/\sqrt{\lambda}$. In this case we take arbitrary partition $\mathcal{A}, \mathcal{B}$ of $\mathcal{H}$. We have:
                    $$w\br{A\cup S_{\mathcal{G}}^*}\cdot w\br{B\cup S_{\mathcal{G}}^*}\geq w\br{ S_{\mathcal{G}}^*}^2 \geq w\br{\mathcal{G}}^2/\lambda.$$
                \item $w\br{ S_{\mathcal{G}}^*} \leq w\br{\mathcal{G}}/\sqrt{\lambda}$.
                For any choice of the partition $\mathcal{A},\mathcal{B}$ of $\mathcal{H}$, we have $\frac{w\br{A\cup B}}{w\br{\mathcal{G}}}\geq 1-\frac{1}{\sqrt{\lambda}}$. We pick $\mathcal{A},\mathcal{B}$ to be a partition of $\mathcal{H}$, such that $w\br{A}\geq w\br{B}\geq \br{\frac{1}{2}-\frac{1}{\sqrt{\lambda}}}\cdot w\br{\mathcal{G}}$ (this is always possible as $\frac{1}{2}-\frac{1}{\sqrt{\lambda}}>0$ and for each $H\in\mathcal{H}$, $w\br{H}\leq w\br{\mathcal{G}}/2$). We have:
                \begin{align*}
                w\br{A\cup S_{\mathcal{G}}^*}\cdot w\br{B\cup S_{\mathcal{G}}^*}&\geq w\br{A}\cdot w\br{B} \\&\geq \br{\br{1-{1}/{\sqrt{\lambda}}}\cdot w\br{\mathcal{G}}-w\br{B}}\cdot w\br{B}\\&\geq  {w\br{\mathcal{G}}^2}/{2}\cdot \br{{1}/{2}-{1}/{\sqrt{\lambda}}} = {w\br{\mathcal{G}}^2}/{\lambda}    
                \end{align*}
                
                
                where the third inequality is by using the fact that the concave function $f\br{w\br{B}}=w\br{B}\cdot\br{\br{1-\frac{1}{\sqrt{\lambda}}}\cdot w\br{\mathcal{G}}-w\br{B}}$ reaches its minimum in the interval $\left[\br{\frac{1}{2}-\frac{1}{\sqrt{\lambda}}}\cdot w\br{\mathcal{G}},w\br{\mathcal{G}}/2\right]$ when $w\br{B}=\br{\frac{1}{2}-\frac{1}{\sqrt{\lambda}}}\cdot w\br{\mathcal{G}}$.
                \end{enumerate}
            \end{proof}
        \end{lemma}

        % \begin{remark}
        %     Note that one can improve the value of the above constant to roughly $\lambda\approx 5.95$, by a more detailed analysis. The prove is deferred to the Appendix \ref{betterconst}.
        % \end{remark}
        
        Using the fact, that the partition $\br{A,S_{\mathcal{G}}^*,B}$ in the above lemma is a vertex cut of $\mathcal{G}$, we have the following upper bound on the optimal value of the min-ratio-vertex cut of $\mathcal{G}$,  $\alpha_{c,w}\br{\mathcal{G}}$:
                $$\alpha_{c,w}\br{\mathcal{G}}\leq \alpha_{c,w}\br{A,S_{\mathcal{G}}^*,B}=\frac{c\br{S_{\mathcal{G}}^*}}{w\br{A\cup S_{\mathcal{G}}^*}\cdot w\br{B\cup S_{\mathcal{G}}^*}}\leq \frac{\lambda\cdot c\br{S_{\mathcal{G}}^*}}{w\br{\mathcal{G}}^2}.
                $$
                
        Let $\br{A_{\mathcal{G}},S_{\mathcal{G}}, B_{\mathcal{G}}}$, be the partition returned by $\FAlgorithmMinCut \br{\mathcal{G},c,w}$. Without the loss of generality assume that $w\br{A_{\mathcal{G}}}\geq w\br{B_{\mathcal{G}}}$. Using Theorem \ref{approxmrvc}, we get that:
        $$
        \alpha_{c,w}\br{A_{\mathcal{G}},S_{\mathcal{G}}, B_{\mathcal{G}}}=\frac{c\br{S_{\mathcal{G}}}}{w\br{A_{\mathcal{G}}\cup S_{\mathcal{G}}}\cdot w\br{B_{\mathcal{G}}\cup S_{\mathcal{G}}}}\leq f_n \cdot \frac{\lambda\cdot c\br{S_{\mathcal{G}}^*}} {w\br{\mathcal{G}}^2}.
        $$
        
        Let $\beta=w\br{B_{\mathcal{G}}\cup S_{\mathcal{G}}}/w\br{\mathcal{G}}$. We have that $\br{1-\beta}\cdot w\br{\mathcal{G}} = w\br{A_\mathcal{G}}$, so we conclude that the contribution of the decision tree $D_{\mathcal{G}}$ is bounded by:
        \begin{align*}
        w\br{\mathcal{G}}\cdot c\br{S_{\mathcal{G}}} &\leq \lambda \cdot f_n\cdot \frac{w\br{A_{\mathcal{G}}\cup S_{\mathcal{G}}}\cdot w\br{B_{\mathcal{G}}\cup S_{\mathcal{G}}}}{w\br{\mathcal{G}}}\cdot c\br{S_{\mathcal{G}}^*}\\
        & \leq 
        \lambda \cdot f_n\cdot w\br{B_{\mathcal{G}}\cup S_{\mathcal{G}}}\cdot c\br{S_{\mathcal{G}}^*} \leq 
        \lambda \cdot f_n\cdot \sum_{k=w\br{A_{\mathcal{G}}}+1}^{w\br{\mathcal{G}}}c\br{S_{\fl{k/2}}^*\cap \mathcal{G}}
        \end{align*}

        where the last inequality is by Lemma \ref{splitting}.

        As before, to bound the cost of the whole solution we will firstly show the following lemma. The argument is mostly the same as for the Lemma \ref{up_trees}, however, there are few differences and we include it for completeness:
            \begin{lemma}\label{up_graphs}
            $$\sum_{\mathcal{G}}\sum_{k=w\br{A_{\mathcal{G}}}+1}^{w\br{\mathcal{G}}}c\br{S_{\fl{k/2}}^*\cap \mathcal{G}}\leq \sum_{k=0}^{w\br{G}}c\br{S_{\fl{k/2}}^*}.$$
            \begin{proof}
                Fix a value of $\mathcal{G}$ and $k$. Their contribution to the cost is $c\br{S_{\fl{k/2}}^*\cap \mathcal{G}}$. Consider which candidate subgraphs contribute such a term. By definition of $S_{\mathcal{G}}$, we have that $\mathcal{G}$ is the minimal subgraph, such that $w\br{\mathcal{G}}\geq k\geq w\br{A_{\mathcal{G}}}+1 >w\br{H}$, for every $H\in \mathcal{G}-S_{\mathcal{G}}$. This means that if for every $H\in \mathcal{G}-S_{\mathcal{G}}$, $w\br{H}<k$, then $\mathcal{G}$ contributes such a term. Since for all $H_1, H_2\in \mathcal{G}-S_{\mathcal{G}}$, $H_1\cap H_2=\emptyset$, we have that $\br{S_{\fl{k/2}}^*\cap H_1}\cup \br{S_{\fl{k/2}}^*\cap H_2}=\emptyset$, the claim follows by summing over all values of $k$.
            \end{proof}
            \end{lemma}
            
        We are now ready to bound the cost of the solution. Let $D$ be the decision tree returned by the procedure. We have:
        \begin{align*}
            $\COST_{D}\br{G}&\leq \sum_{\mathcal{G}}w\br{\mathcal{G}}\cdot c\br{S_{\mathcal{G}}}\\
            &\leq \lambda \cdot f_n\cdot \sum_{\mathcal{G}}\sum_{k=w\br{A_{\mathcal{G}}}+1}^{w\br{\mathcal{G}}}c\br{S_{\fl{k/2}}^*\cap \mathcal{G}}\leq 
            \lambda \cdot f_n\cdot\sum_{k=0}^{w\br{G}}c\br{S_{\fl{k/2}}^*} \\
            &\leq 2\cdot \lambda \cdot f_n\cdot \OPT\br{G} = \br{12+4\sqrt{5}}\cdot f_n \cdot \OPT\br{G}
        \end{align*}
        
    where the third inequality is due to Lemma \ref{up_graphs} and the last inequality is by Lemma \ref{lb_opt}.
    \end{proof}

\end{theorem}
    

