\SetKwFunction{FRankingBasedDT}{RankingBasedDT}
\SetKwFunction{FQPTAS}{QPTAS}
\SetKwFunction{FBuildDT}{BuildDT}
\newpage
\section{Trees, worst case, non-uniform costs}
% In this section we will be only concerned with the vertex-query variant of the problem. This is due to the fact that the edge variant is easily reducible to the vertex variant of the problem and this reduction preserves the approximation ratio (note that this reduces the problem to the version in which the last query may be omitted but all of the algorithms can be easily altered to consider this assumption). This is done by subdividing each edge $e$ with a new vertex $v_e$ of cost $c\br{v_e}=c\br{e}$. If we consider the average case criterion then we additionally set $w\br{v_e}=0$. It is immediate that the optimal decision tree for the this vertex query model instance can be used to obtain a decision tree for the original instance. To do so, simply replace each query to vertex $v_e$ with the query corresponding to $e$. 

The problem for non-uniform is NP-hard even when restricted to spiders of diameter $6$ and binary trees.
A simple greedy heuristics which always queries the middle vertex of the graph achieves a $O\br{\log n}$-approximation \cite{Dereniowski2009ERankOfWTs}. However one can obtain better results. 
We begin with the following simple lemma, which will become useful in few arguments:
\begin{lemma}\label{lemma:subtreeCost}
    Let $T'$ be a connected subtree of $T$. Then, $\OPT\br{T'}\leq\OPT\br{T}$.
\end{lemma}

\subsubsection{Vertex ranking}\label{vertexRanking}
The \textit{vertex ranking} of $T$ is a labeling of vertices $l:V\to \brc{1,2,\dots,\cl{\log n}+1}$, which satisfies the following condition: for each pair of vertices $u,v\in V\br{T}$, whenever $l\br{u}=l\br{v}$, there exists $z\in \mathcal{P}_T\br{u,v}$ for which $l\br{z}>l\br{v}$. Such a labeling always exists and can be computed in linear time by means of dynamic programming \cite{Schaffer1989OptNodeRankOfTsInLinTime, OnakParys2006GenOfBSSInTsAndFLikePosets, Mozes_Onak2008FindOptTSStartInLinTime}. For a visual example, see Figure \ref{vertex_ranking_figure}

\begin{figure}[htbp]
    \centering
    \begin{minipage}{0.26\textwidth}
        \centering
        \tikz [tree layout, grow=-65,
               sibling distance=7mm, level distance=11mm,]
          \graph {
    ""[as=$a$] -- {
        ""[as=$b$] -- ""[as=$c$] -- {
            ""[as=$d$] -- ""[as=$e$],
            ""[as=$f$] -- { ""[as=$g$], ""[as=$h$], ""[as=$i$] }
        },
        ""[as=$j$] -- ""[as=$k$] -- { ""[as=$l$], ""[as=$m$] }
    }
};
    \caption[Sample tree with uniform costs]{}
    \label{vr_tree}
    \end{minipage}
    \begin{minipage}{0.26\textwidth}
        \centering
        \tikz [tree layout, grow=-65,
               sibling distance=7mm, level distance=11mm,]
          \graph {
    ""[as=$4$] -- {
        ""[as=$1$] -- ""[as=$3$] -- {
            ""[as=$2$] -- ""[as=$1$],
            ""[as=$2$] -- { ""[as=$1$], ""[as=$1$], ""[as=$1$] }
        },
        ""[as=$1$] -- ""[as=$2$] -- { ""[as=$1$], ""[as=$1$] }
    }
};
    \caption[Sample tree with uniform costs]{}
    \label{vr_ranking}
    \end{minipage}
    \begin{minipage}{0.45\textwidth}
        \centering
        \tikz [tree layout, grow=-90,
               sibling distance=8mm, level distance=15.5mm,]
            \graph {
              ""[as=$a$] -> {
                ""[as=$c$] -> {""[as=$b$], ""[as=$d$] -> {""[as=$e$]}, ""[as=$f$] -> {""[as=$g$], ""[as=$h$], ""[as=$i$]}}, ""[as=$k$]->{""[as=$j$], ""[as=$l$], ""[as=$m$]}
              }
            };
        \caption[Edge decision tree for a tree]{}
        \label{vr_dt}
    \end{minipage}
        \caption[Tree and decision trees for it]{Sample input tree $T$ (Figure \ref{vr_tree}), vertex ranking labeling $l$ of $T$ (Figure \ref{vr_ranking}) and a decision tree $D$ for $T$ built using $l$ (Figure \ref{vr_dt}).}
        \label{vertex_ranking_figure}
\end{figure}

Having a vertex ranking of $T$, one can easily obtain a decision tree for $T$ using the following procedure:
\begin{enumerate}
    \item Let $z\in V\br{T}$ be the unique vertex, such that for every $v\in V\br{T}$, $l\br{z}\geq l\br{v}$.
    \item Schedule a query to $z$ as the root of the decision tree $D$ for $T$.
    \item For each $T'\in T-z$, build a decision tree $D_{T'}$ recursively and hang it below the query to $z$ in $D$.
\end{enumerate} 

When the input tree has uniform costs and the ranking uses the minimal number of labels, the decision tree built in this way is optimal and never uses more than $\fl{\log n} + 1$ queries \cite{OnakParys2006GenOfBSSInTsAndFLikePosets}. Let \FRankingBasedDT be the name of the latter procedure. We have the following corollary:
  
\begin{corollary}\label{vertexRankingCorollary}
    There exists an $O\br{n}$ time procedure \FRankingBasedDT that finds the optimal decision tree for the Tree Search Problem when all costs are uniform. Moreover, the depth of such a decision tree, i.e., the worst-case number of queries, is at most $\fl{\log n} + 1$.
\end{corollary}

\subsection{A warm up: $O\br{\log n/\log\log n}$-approximation algorithm  for $T||V,c||C_{max}$}
This first algorithm is an adapted and simplified version of the algorithm due to \cite{Cicalese2016OnTSPwNonUniCost} for the edge query model.
\begin{theorem}
    There exists a polynomial time, $O\br{\log n/\log\log n}$-approximation algorithm for the $T||V,c||C_{max}$ problem .
    \begin{proof}
        
To construct a decision tree we will use the following exact procedure:
\begin{lemma}
    There exists a $O\br{2^nn}$ algorithm for $T||V,c||C_{max}$
    \begin{proof}
        The algorithm is a general version of the dynamic programming procedure for paths. We have that:
        $$
        \OPT_{max}\br{T} = \min_{v\in V\br{T}}\brc{c\br{v}+\max_{H\in T-v}\brc{\OPT_{max}\br{H}}}
        $$
        There are there are at most $O\br{2^n}$ different subtrees of $T$ to be checked. Additionally, and for each $v\in V\br{T}$ there are at most $\deg_T\br{v}$ possible responses to check in the inner $\max$ function. Therefore for each subproblem there are at most
        $
        \sum_{v\in V\br{T}}\deg_T\br{v} = 2m = 2n-2
        $
        comparison operations to be performed. As at each level of the recursion the algorithm considers all possible choices of the next queried vertex $v$ it returns the optimal decision tree for $T$ and the claim follows.
    \end{proof}
\end{lemma}
\begin{observation}\label{neighborsPathObservation}
    Let $D$ be a partial decision tree for tree $T$. Let $T'$ be a subtree of $T$. Let $Q$ be a set of all queries to vertices from $N_{T}\br{V\br{T'}}$ in $D$ such that every for every $q\in Q$: $q$ is queried before every vertex in $T'$. Then $D\angl{Q}$ is a path. 
    \begin{proof}
        Let $x\in N_{T}\br{V\br{T'}}$. In such case, for every vertex $v\in V\br{T-V\br{T'}-N_{T}\br{V\br{T'}}}$ the answer to a query to $v$ is always towards the same $u\in N_{T}\br{v}$, so until a vertex from $T'$ is queried, no query $q$ can partition vertices from $N_{T}\br{V\br{T'}}$ into disjoint subtrees of candidate vertices except when $q\in N_{T}\br{V\br{T'}}$. After a query to $q$, the only different response is when $x=q$, in which case no further queries are needed, so queries in $Q$ must belong to a path in $D$.
    \end{proof}
\end{observation}



Let $k=2^{\fl{\log\log n}+2}$.
The basic idea is as follows. The algorithm is recursive. Let $\mathcal{T}$ be the tree currently processed by the algorithm. If $n\br{T}\leq k$ then we use the exponential time algorithm to find the optimal solution in time $2^kk=\text{poly}\br{n}$.

If otherwise, to build a solution (see Algorithm \ref{cicaleseInspired}) we will firstly define a set $\mathcal{X}\subseteq V\br{\mathcal{T}}$ which will be of size at most $k$. We build $\mathcal{X}$ iteratively. Starting with an empty set we pick the centroid $x_1$ of $T$ which we add to $\mathcal{X}$. Then we take the forest $F=T-x$, find the largest $H\in F$, pick its centroid $x_2$ and append it to $\mathcal{X}$. We continue this in $F-H + \br{H-x_2}$ until $\spr{\mathcal{X}}=k$.
\begin{lemma}\label{lemma:componentSize}
    For every $H\in \mathcal{T}-\mathcal{X}$ we have that $n\br{H}\leq n\br{\mathcal{T}}/\log\br{n}$.
    \begin{proof}
        We prove by induction on $t$ that deleting first $2^t$ centroids from $T$ each connected components $H_t$ has size at most $n\br{H_t}\leq n\br{\mathcal{T}}/2^{t-1}$. For the case when $t=0$ we have that after 1 iteration every $H_1$ has size at most $n\br{T}/2\leq 2\br{n}$ so the base of induction is complete.

        Fix $t>0$ and by assume by the induction hypothesis that after $2^{t-1}$ iterations all
    \end{proof}
\end{lemma}

We also define set $\mathcal{Y}\subseteq V\br{\mathcal{T}}$ which consists of vertices in $\mathcal{X}$ and all vertices in $v\in \mathcal{T}\angl{X}$ such that $\deg_{T\angl{\mathcal{X}}}\br{v}\geq 3$.
Furthermore, we define set $\mathcal{Z}\subseteq V\br{\mathcal{T}}$ as a set consisting of vertices in $\mathcal{Y}$ and for every $u,v\in \mathcal{Y}$ such that $\mathcal{P}_{\mathcal{T}}\br{u, v}\neq\emptyset$ and $\mathcal{P}_{\mathcal{T}}\br{u, v}\cap \mathcal{Y}=\emptyset$ we add to $\mathcal{Z}$ the vertex $\argmin_{z\in \mathcal{P}_{\mathcal{T}}\br{u, v}}\brc{c\br{z}}$ (for example see Figure \ref{exampleTreeWithSetZ}). We then create an auxiliary tree $\mathcal{T}_{\mathcal{Z}}=\br{\mathcal{Z},\brc{uv|\mathcal{P}_{\mathcal{T}}\br{u, v}\cap \mathcal{Z}=\emptyset}}$ (for example see Figure \ref{exampleAuxTreeTZ}). The algorithm builds an optimal decision tree $D_{\mathcal{Z}}$ for $\mathcal{T}_{\mathcal{Z}}$ by applying the exponential time algorithm. Observe, that  $D_{\mathcal{Z}}$ is a  partial decision tree for $\mathcal{T}$, so we get that:
\begin{observation}\label{observation:CostDZinT}
    $\COST_{D_{\mathcal{Z}}}\br{\mathcal{T}_{\mathcal{Z}}}=\COST_{D_{\mathcal{Z}}}\br{\mathcal{T}}$.
\end{observation}
Then for each $H\in \mathcal{T}-\mathcal{Z}$ we recursively apply the same algorithm to obtain the decision tree $D_H$ and we hang it in $D_\mathcal{Z}$ below the unique last query to vertex in $N_{\mathcal{T}'}\br{H}$ (By Observation \ref{neighborsPathObservation}).

\begin{algorithm}
\caption{Main recursive procedure ($k$ is a global parameter)}\label{cicalese_inspired_pseudocode}
\SetKwFunction{FDecisionTree}{DecisionTree}
\SetKwProg{Fn}{Procedure}{:}{}
\Fn{\FDecisionTree{$\mathcal{T},c$}}{
    \If{$n\br{\mathcal{T}}\leq k$}{
    $D\gets$\FExact{$\mathcal{T},c$}.
    
    \Return $D$
    }
    $\mathcal{X}\gets\emptyset$.
    
    $\mathcal{F}\gets\brc{\mathcal{T}}$.
    
    \For{$1\leq i\leq k$ } 
        {
        \If{$\mathcal{F}=\emptyset$}{
            \textbf{break}
        }
        $H\gets\argmax_{H\in \mathcal{F}}\brc{n\br{H}}$.
        
        $x\gets$ the centroid of $H$.
        
        $\mathcal{X}\gets \mathcal{X}\cup\brc{x}$.
        
        $\mathcal{F}\gets \mathcal{F} \cup H-x$.
        }
    $\mathcal{Z}\gets\mathcal{Y}\gets\mathcal{X}\cup \brc{v\in \mathcal{T}\angl{\mathcal{X}}|\deg_{\mathcal{T}\angl{\mathcal{X}}}\br{v}\geq3}$.
    \tcp{Branching vertices in $\mathcal{T}\angl{X}$.}

    \ForEach{$u,v\in \mathcal{Y},\mathcal{P}_{\mathcal{T}}\br{u, v}\neq\emptyset,\mathcal{P}_{\mathcal{T}}\br{u, v}\cap \mathcal{Y}=\emptyset$}
    {
        $\mathcal{Z}\gets \mathcal{Z}\cup\brc{\argmin_{z\in \mathcal{P}_{\mathcal{T}}\br{u, v}}\brc{c\br{z}}}$. 
        \tcp{Lightest vertex on path $P_\mathcal{T}\br{u,v}$.}
    }
    
    $\mathcal{T}_{\mathcal{Z}}=\br{\mathcal{Z}, \brc{uv|\mathcal{P}_{\mathcal{T}}\br{u, v}\cap \mathcal{Z}=\emptyset}}$.
    
    $D\gets D_{\mathcal{Z}}\gets $\FExact{$\mathcal{T}_{\mathcal{Z}},c$}.

    \ForEach{$H\in \mathcal{T}-\mathcal{Z}$}
    {
        $D_H\gets$\FDecisionTree{$H,c$}.
        
        Hang $D_H$ in $D$ below the last query to a vertex $v \in N_{\mathcal{T}}\br{H}$.
    }
        
    \Return $D$
    
}
\end{algorithm}

\begin{lemma}\label{lemma:auxTreeSize}
    Let $\mathcal{T}_{\mathcal{Z}}$ be the auxiliary tree. Then, $\spr{V\br{\mathcal{T}_{\mathcal{Z}}}}\leq 4k-3$.
    \begin{proof}
        We firstly show that $\spr{\mathcal{Y}}\leq 2k-1$. We use induction of the centroids in $\mathcal{X}$. For $1\leq i\leq k$ let $x_i$ denote the $i$-th centroid added to $\mathcal{X}$. We will construct a family of sets $\mathcal{X}_1, \mathcal{X}_2,\dots, \mathcal{X}_{\spr{\mathcal{H}}}$ such that for any $1\leq t\leq \spr{\mathcal{X}}$: $\spr{\mathcal{X}_t}=t$ and $\mathcal{X}_{\spr{\mathcal{X}}}=\mathcal{X}$. For each $\mathcal{X}_t$ we will also construct a corresponding set $\mathcal{Y}_t$, ensuring $\mathcal{Y}_{\spr{\mathcal{X}}}=\mathcal{Y}$. We will build the sets $\mathcal{Y}_{t}$ to ensure that $\spr{\mathcal{Y}_t}\leq 2t-1$. 
        
        Let $\mathcal{X}_1=\brc{x_1}$, $\mathcal{Y}_1=\brc{x_1}$. This establishes the base case. Assume by induction on $t\geq1$ that $\spr{\mathcal{Y}_t}\leq 2t-1$ for some $t>1$. Let $\mathcal{X}_{t+1} =\mathcal{X}_{t}\cup \brc{x_{t+1}}$ and let $\mathcal{T}_t=\mathcal{T}\angl{\mathcal{X}_{t}}$ If $x_t\in V\br{\mathcal{T}_t}$ then $\mathcal{Y}_{t+1}=\mathcal{Y}_{t}\cup \brc{x_t}$. If otherwise let $y_t \in V\br{\mathcal{T}_t}$ be the unique vertex such that $P\br{x_t, y_t}\cap V\br{\mathcal{T}_t}=\emptyset$. Then $\mathcal{Y}_{t+1}=\mathcal{Y}_{t}\cup \brc{x_t, y_t}$. As by induction $\spr{\mathcal{Y}_{t}}\leq 2t-1$ and we add at most two vertices to it to obtain $\mathcal{Y}_{t+1}$ the induction step is complete.
        
        As paths between vertices in $\mathcal{Y}$ form a tree, at most $2k-2$ additional vertices are added to $\mathcal{Y}$ while constructing $\mathcal{Z}$ (at most one for each path) and the lemma follows.
    \end{proof}
\end{lemma}
\begin{lemma}\label{lemma:auxTreeCost}
    Let $\mathcal{T}_{\mathcal{Z}}$ be the auxiliary tree. Then, $\OPT\br{\mathcal{T}_{\mathcal{Z}}}\leq \OPT\br{\mathcal{T}}$.
    \begin{proof}
        Let $D^*$ be the optimal strategy for $\mathcal{T}\angl{\mathcal{Z}}$. We build a new decision tree $D_{\mathcal{Z}}'$ for $\mathcal{T}_{\mathcal{Z}}$ by transforming $D^*$: Let $u,v\in \mathcal{Y}$ such that $\mathcal{P}_{\mathcal{T}}\br{u, v}\neq\emptyset$ and $\mathcal{P}_{\mathcal{T}}\br{u, v}\cap \mathcal{Y}=\emptyset$. Let $q\in V\br{D^*}$ such that $q\in \mathcal{P}_{\mathcal{T}}\br{u, v}$ is the first query among vertices of $\mathcal{P}_{\mathcal{T}}\br{u, v}$. We replace $q$ in $D^*$ by the query to the distinct vertex $v_{u,v}\in \mathcal{P}_{\mathcal{T}}\br{u, v}\cap \mathcal{Z}$ and delete all queries to vertices $\mathcal{P}_{\mathcal{T}}\br{u, v}-v_{u,v}$ from $D^*$. By construction, $D_{\mathcal{Z}}'$ is a valid decision tree for $\mathcal{T}_{\mathcal{Z}}$ and as for every $z\in \mathcal{P}_{\mathcal{T}}\br{u, v}$: $c\br{v_{u,v}}\leq c\br{z}$ such strategy has cost at most $\COST_{D_{\mathcal{Z}}'}\br{\mathcal{T}_{\mathcal{Z}}}\leq \OPT\br{\mathcal{T}\angl{\mathcal{Z}}}$. We get:
        $$
        \OPT\br{\mathcal{T}_{\mathcal{Z}}}\leq \COST_{D_{\mathcal{Z}}'}\br{\mathcal{T}_{\mathcal{Z}}}\leq \OPT\br{\mathcal{T}\angl{\mathcal{Z}}}\leq \OPT\br{\mathcal{T}}
        $$

        where the first inequality is due to the optimality and the last inequality is due to the fact that $\mathcal{T}\angl{\mathcal{Z}}$ is a subtree of $\mathcal{T}$ (by Lemma \ref{lemma:subtreeCost}). The lemma follows.
    \end{proof}
\end{lemma}
\begin{lemma}
    Let $D_T$ be the solution returned by the algorithm. Then the approximation factor of such solution is bounded by 
    $
    \APP_T\br{D_T}\leq \log n/\log\log n
    $.
    \begin{proof}
        Let $\mathcal{T}$ be the tree processed at some level of the recursion and let $D_{\mathcal{T}}$ be the decision tree returned by the algorithm. The proof is by induction on the size of $\mathcal{T}$.  We claim that $\APP_{\mathcal{T}}\br{D_{\mathcal{T}}}\leq \max\brc{1, \log n\br{\mathcal{T}}/\log\log n}$. If $n\br{\mathcal{T}}\leq k$ then $D_{\mathcal{T}}$ is the optimal decision tree for $\mathcal{T}$ which establishes the base case. Let $n\br{\mathcal{T}} > k$ and assume that claim holds for every $t< n\br{\mathcal{T}}$. 
        By construction, we have that:
        \begin{align*}
        \APP_{D_{\mathcal{T}}}\br{\mathcal{T}}&=\frac{\COST_{D_{\mathcal{T}}}\br{\mathcal{T}}}{\OPT\br{\mathcal{T}}}\\
        &\leq \frac{\COST_{D_{\mathcal{Z}}}\br{\mathcal{T}}+\max_{H\in \mathcal{T}-\mathcal{Z}}\brc{C_{D_H}\br{H}}}{\OPT\br{\mathcal{T}}}\\
        &\leq 
        \frac{\COST_{D_{\mathcal{Z}}}\br{\mathcal{T}_{\mathcal{Z}}}}{\OPT\br{\mathcal{T}_{\mathcal{Z}}}}+\max_{H\in \mathcal{T}-\mathcal{Z}}\brc{\frac{C_{D_H}\br{H}}{\OPT\br{H}}}\\
        &\leq 1+\frac{\log\frac{n\br{\mathcal{T}}}{\log n\br{\mathcal{T}}}}{\log\log n}= \frac{\log n\br{\mathcal{T}}}{\log\log n}
        \end{align*}
        where the first inequality is by construction, the second is by usage of Observation \ref{observation:CostDZinT}, Lemma \ref{lemma:auxTreeCost} and Lemma \ref{lemma:subtreeCost} and the last inequality is due to the Lemma \ref{lemma:componentSize} and the induction hypothesis.
    \end{proof}
\end{lemma}
    Using the fact that the call to the exponential time procedure requires $O\br{2^{4k-3}\br{4k-3}}=\text{poly}\br{n}$ time (Due to Lemma \ref{lemma:auxTreeSize}), all other computations require polynomial time, and each $v\in V\br{T}$ belongs to $\mathcal{Z}$ at most once during the execution we get that the overall running time is polynomial in $n$.
    \end{proof}
\end{theorem}
In the above analysis we lose one factor of $\OPT$ per each level of recursion of which there are at most $O\br{\log n/\log\log n}$. Notice however, that we can allow some more loss (i. e. $c\cdot\OPT$) without affecting the asymptotical approximation factor. As it turns out it is possible to obtain a constant factor approximation for this problem in quasipolynomial time. This is the main idea behind the improvement of the approximation factor for this problem as in such case the size of the set $\mathcal{Z}$ may be greater and less recursion levels are needed which directly improves the approximation.

\subsection{An $O\br{\sqrt{\log n}}$-approximation algorithm for $T||V,c||C_{max}$}
We begin with the following proposition \cite{dereniowski2017ApproxSsForGeneralBSinWTs} about the existence of QPTAS for $T||V,c||C_{max}$:
\begin{proposition}\label{QPTAS}
     For any $0 < \epsilon \leq 1$ there exists a $(1+\epsilon)$-approximation algorithm for the Tree Search Problem running in $2^{O\br{\frac{\log^2n}{\epsilon^2}}}$ time.
\end{proposition}
The algorithm and the proof of its correctness are very intricate and requires usage of an alternative notion of strategy. However, we rewrite it to use the language of the decision trees. Since the proof is involved for now we will use it as a black-box. The proof will be differed to a separate paragraph after the analysis below.

\begin{theorem}
    There exists a polynomial time, $O\br{\log n/\log\log n}$-approximation algorithm for the $T||V,c||C_{max}$ problem .
    \begin{proof}
    We use the same procedure as in the $O\br{\log n/\log\log n}$-approximation algorithm, however we set $k=2^{\fl{\sqrt{\log n}}+2}$ and we swap the exact procedure to the QPTAS with $\epsilon=1$. The analysis of the algorithm is largely the same, except while evaluating the cost of the resulting decision tree.
    \begin{lemma}
        Let $D_T$ be the solution returned by the algorithm. Then the approximation factor of such solution is bounded by 
    $
    \APP_T\br{D_T}\leq 2\sqrt{\log n}
    $.
    \begin{proof}
        Let $\mathcal{T}$ be the tree processed at some level of the recursion and let $D_{\mathcal{T}}$ be the decision tree returned by the algorithm. The proof is by induction on the size of $\mathcal{T}$.  We claim that $\APP_{\mathcal{T}}\br{D_{\mathcal{T}}}\leq \max\brc{1, 2\log n\br{\mathcal{T}}/\sqrt{\log n}}$. If $n\br{\mathcal{T}}\leq k$ then $D_{\mathcal{T}}$ is the optimal decision tree for $\mathcal{T}$ which establishes the base case. Let $n\br{\mathcal{T}} > k$ and assume that claim holds for every $t< n\br{\mathcal{T}}$. 
        By construction, we have that:
        \begin{align*}
        \APP_{D_{\mathcal{T}}}\br{\mathcal{T}}&=\frac{\COST_{D_{\mathcal{T}}}\br{\mathcal{T}}}{\OPT\br{\mathcal{T}}}\\
        &\leq \frac{\COST_{D_{\mathcal{Z}}}\br{\mathcal{T}}+\max_{H\in \mathcal{T}-\mathcal{Z}}\brc{C_{D_H}\br{H}}}{\OPT\br{\mathcal{T}}}\\
        &\leq \frac{\COST_{D_{\mathcal{Z}}}\br{\mathcal{T}_{\mathcal{Z}}}}{\OPT\br{\mathcal{T}_{\mathcal{Z}}}}+\max_{H\in \mathcal{T}-\mathcal{Z}}\brc{\frac{C_{D_H}\br{H}}{\OPT\br{H}}}\\
        &\leq 2+\frac{2\log\br{\frac{n\br{\mathcal{T}}}{\sqrt{\log n}}}}{\sqrt{\log n}}= \frac{2\log n\br{\mathcal{T}}}{\sqrt{\log n}}
        \end{align*}
        
        where the first inequality is by construction, the second is by usage of Observation \ref{observation:CostDZinT}, Lemma \ref{lemma:auxTreeCost} and Lemma \ref{lemma:subtreeCost} and the last inequality is due to the Lemma \ref{lemma:componentSize} and the induction hypothesis.
    \end{proof}
\end{lemma}
    \end{proof}
\end{theorem}


\begin{algorithm}
\caption{The QPTAS for $T||V,c,w||C_{max}$.}\label{qptas_pseudocode}
\SetKwProg{Fn}{Procedure}{:}{}
\Fn{$\FQPTAS\br{T,c,\epsilon}$}{
    $p\gets\cl{24/\epsilon}$,
    $k\gets 0$,
    $D\gets \emptyset$.

    $d\gets p^2\cdot\br{\fl{\log\br{n}}+1}$.

    \Repeat{$D \neq \emptyset$}
    {
        $k\gets k+\frac{1}{pn}$.

        \ForEach{$v\in V\br{T}$}{
            \If{$c\br{v}>pk$}{
                $c'\br{v}\gets \cl{c\br{v}}_k$.
            }
            \Else{
                $c'\br{v}\gets \cl{c\br{v}}_{\frac{1}{pn}}$.
            }
            
        }

        $T_C\gets T$ with all heavy modules contracted.

        $D_C\gets\FRankingBasedDT\br{T_C}$.

        $D\gets \FBuildDT\br{T, c', D_C, p, k, d}$.
    }   
    \Return $D$.
    
}
\end{algorithm}

\SetKwFunction{FCreateDecisionTree}{CreateDecisionTree}

\subsection{$O\br{\log\log n}$-approximation algorithm parametrized by the $k$-up-modularity of the cost function}
\subsubsection{$k$-up-modularity}\label{kUpModularity}
The main algorithmic difficulty in dealing with the problem arises when the values of the cost function vary drastically. We would like to measure this "irregularity" in a quantifiable way. To do so, we introduce the notion of $k$-up-modularity.


\begin{figure}[htbp]
\centering
\usetikzlibrary{3d}
\usetikzlibrary{calc}

% Isometric projection setup
\def\angi{27}
\def\angii{62}
\pgfmathsetmacro\xx{sin(\angii)}
\pgfmathsetmacro\xy{-cos(\angii)*sin(\angi)}
\pgfmathsetmacro\yx{sin(\angii-90)}
\pgfmathsetmacro\yy{-cos(\angii-90)*sin(\angi)}
\pgfmathsetmacro\zx{0}
\pgfmathsetmacro\zy{cos(\angi)}


\begin{tikzpicture}[ scale=2,
  x={({\xx cm,\xy cm})},
  y={({\yx cm,\yy cm})},
  z={({\zx cm,\zy cm})},
  line cap=round, line join=round]


\begin{scope}[canvas is xy plane at z=1.05]
    \fill[yellow!20!orange!20, opacity=0.5] (-0.1,-0.1) rectangle (5.5,3.8);
    \draw[yellow!50!orange, very thick] (-0.1,-0.1) rectangle (5.5,3.8);
    \node[anchor=north west, font=\small\bfseries, yellow!40!orange] at (0.1,0.1) {$s$};
  \end{scope}
  % Axes
  \draw[->, very thick] (0,0,0) -- (5.5,0,0) node[anchor=north east] {};
  \draw[->, very thick] (0,0,0) -- (0,4,0) node[anchor=north west] {};
  \draw[->, very thick] (0,0,0) -- (0,0,1.75) node[anchor=south] {$c\br{v}$};

  % Node visually aligned with the xy-plane
    \vertex{1}{0.25,0.25}{0.0}{}{};
    \vertex{2}{0.7,0.45}{0.0}{}{};
    \vertex{3}{1.25,0.75}{0.0}{}{};
    \vertex{4}{1.7,1.15}{0.0}{}{};
    
    \vertex{5}{1.55,1.7}{0.0}{}{};
    \vertex{6}{1.5,2.3}{0.0}{}{};
    \vertex{7}{1.4,3}{0.0}{}{};
    \vertex{8}{0.9,3.3}{0.0}{}{};
    \vertex{9}{0.3,3.5}{0.0}{}{};
    
    \vertex{10}{2.75,0.75}{0.0}{}{};
    \vertex{11}{2.75,1.4}{0.0}{}{};
    \vertex{12}{2.5, 1.75}{0.0}{}{};
    
    \vertex{13}{3.2,0.6}{0.0}{}{};
    \vertex{14}{3.75,0.6}{0.0}{}{};
    \vertex{15}{4.2,0.5}{0.0}{}{};
    \vertex{16}{5,0.4}{0.0}{}{};

  \draw[very thick] (1) to (2);
  \draw[very thick] (2) to (3);
  \draw[very thick] (3) to (4);
  \draw[very thick] (4) to (5);
  \draw[very thick] (5) to (6);
  \draw[very thick] (6) to (7);
  \draw[very thick] (7) to (8);
  \draw[very thick] (8) to (9);
  
  \draw[very thick] (4) to (10);
  \draw[very thick] (10) to (11);
  \draw[very thick] (11) to (12);
  
  \draw[very thick] (10) to (13);
  \draw[very thick] (13) to (14);
  \draw[very thick] (14) to (15);
  \draw[very thick] (15) to (16);


\vertex{91}{0.25,0.25}{0.35}{}{green!50!black, dotted};
\vertex{92}{0.7,0.45}{0.9}{}{green!50!black, dotted};
\vertex{93}{1.25,0.75}{1.75}{}{blue};
\vertex{94}{1.7,1.15}{1.5}{}{blue};

\vertex{95}{1.55,1.7}{1.25}{}{blue};
\vertex{96}{1.5,2.3}{0.95}{}{green!50!black, dotted};
\vertex{97}{1.4,3}{0.8}{}{green!50!black, dotted};
\vertex{98}{0.9,3.3}{1.65}{}{blue};
\vertex{99}{0.3,3.5}{1.25}{}{blue};

\vertex{910}{2.75,0.75}{1.75}{}{blue};
\vertex{911}{2.75,1.4}{1.65}{}{blue};
\vertex{912}{2.5, 1.75}{0.8}{}{green!50!black, dotted};

\vertex{913}{3.2,0.6}{1.65}{}{blue};
\vertex{914}{3.75,0.6}{1.35}{}{blue};
\vertex{915}{4.2,0.5}{0.95}{}{green!50!black, dotted};
\vertex{916}{5,0.4}{0.7}{}{green!50!black, dotted};

  
  \draw[->, green!50!black, dotted, dashed] (1) to (91);
  \draw[->, green!50!black, dotted, dashed] (2) to (92);
  \draw[green!50!black, dotted, dashed] (3) to (1.25,0.75, 1.05);
  \draw[->, blue, dashed] (1.25,0.75, 1.05) to (93);
  
  \draw[green!50!black, dotted, dashed] (4) to (1.7,1.15, 1.05);
  \draw[->, blue, dashed] (1.7,1.15, 1.05) to (94);
  
  
  \draw[green!50!black, dotted, dashed] (5) to (1.55,1.7, 1.05);
  \draw[->, blue, dashed] (1.55,1.7, 1.05) to (95);
  
  \draw[blue, dashed] (6) to (1.5,2.3, 0.19);
  \draw[->, green!50!black, dotted, dashed] (1.5,2.3, 0.19) to (96);
  
  \draw[blue, dashed] (7) to (1.4,3, 0.6);
  \draw[->, green!50!black, dotted, dashed] (1.4,3, 0.6) to (97);
  
  \draw[blue, dashed] (8) to (0.9,3.3, 0.77);
  \draw[green!50!black, dotted, dashed] (0.9,3.3, 0.77) to (0.9,3.3, 1.05);
  \draw[->, blue, dashed] (0.9,3.3, 1.05) to (98);
  
  
  \draw[blue, dashed] (9) to (0.3,3.5, 0.88);
  \draw[green!50!black, dotted, dashed] (0.3,3.5, 0.88) to (0.3,3.5, 1.05);
  \draw[->, blue, dashed] (0.3,3.5, 1.05) to (99);
  
  
  \draw[green!50!black, dotted, dashed] (10) to (2.75,0.75, 1.05);
  \draw[->, blue, dashed] (2.75,0.75, 1.05) to (910);
  ;
  
  \draw[green!50!black, dotted, dashed] (11) to (2.75,1.4, 1.05);
  \draw[->, blue, dashed] (2.75,1.4, 1.05) to (911);
  
  \draw[->, green!50!black, dotted, dashed] (12) to (912);
  
  \draw[green!50!black, dotted, dashed] (13) to (3.2,0.6, 1.05);
  \draw[->, blue, dashed] (3.2,0.6, 1.05) to (913);
  
  
  \draw[green!50!black, dotted, dashed] (14) to (3.75,0.6, 1.05);
  \draw[->, blue, dashed] (3.75,0.6, 1.05) to (914);
  
  \draw[->, green!50!black, dotted, dashed] (15) to (915);

  \draw[blue, dashed] (16) to (5,0.4, 0.51);
  \draw[->, green!50!black, dotted , dashed] (5,0.4, 0.51) to (916);

    \draw[very thick, green!50!black, dotted] (91) to (92);
  
  \draw[very thick, green!50!black, dotted] (92) to (0.8,0.5, 1.05);
  \draw[very thick, blue] (0.8,0.5, 1.05) to (93);
  
  \draw[very thick, blue] (93) to (94);
  \draw[very thick, blue] (94) to (95);
  
  \draw[very thick, blue] (95) to (1.55,2.25, 1.05);
  \draw[very thick, green!50!black, dotted] (1.55,2.25, 1.05) to (96);
  
  \draw[very thick, green!50!black, dotted] (96) to (97);

  \draw[very thick, green!50!black, dotted] (97) to (1.23,3.07, 1.05);
  \draw[very thick, blue] (1.23,3.07, 1.05) to (98);
  
  \draw[very thick, blue] (98) to (99);
  
  \draw[very thick, blue] (94) to (910);
  \draw[very thick, blue] (910) to (911);
  
  \draw[very thick, blue] (911) to (2.59, 1.7, 1.05);
  \draw[very thick, green!50!black, dotted] (2.59, 1.7, 1.05) to (912);
  
  \draw[very thick, blue] (910) to (913);
  \draw[very thick, blue] (913) to (914);
  
  \draw[very thick, blue] (914) to (4.08,0.52, 1.05);
  \draw[very thick, green!50!black, dotted] (4.08,0.52, 1.05) to (915);
  
  \draw[very thick, green!50!black, dotted] (915) to (916);





\end{tikzpicture}
\caption[$k$-up modularity]{A visual depiction of a tree $T$ with a $3$-up-modular cost function $c$. Each vertex of a tree is mapped onto some value of $c$. The yellow plane represents some threshold value $t\in \mathbb{R}_{\geq 0}$ (in this particular example $k\br{T,c,t}=2$). The (two) blue subtrees represent members of $\mathcal{H}_{T,c}\br{t}$.}\label{kUpModularityExample}

\end{figure}


Let $t\in\mathbb{R}_{\geq0}$. We define a \textit{heavy module} with respect to $t$ as $H\subseteq V\br{T}$ such that, $T[H]$ is connected, for every $v \in H$, $c\br{v} > t$, and $H$ is maximal - no vertex can be added to it without violating one of its properties. We then define the \textit{heavy module set} with respect to $t$ in $\br{T,c}$ as:
$$
\mathcal{H}_{T,c}\br{t}=\brc{H\subseteq V\br{T}\mid H \text{ is a heavy module w.r.t. } t},
$$

Let $k\br{T,c, t} = \spr{\mathcal{H}_{T,c}\br{t}}$ be the size of the heavy module set, and finally let $k\br{T,c}= \max_{s\in\mathbb{R}_{\geq 0}}\brc{k\br{T,c, t}}$. We say that a function $c$ is $k$\textit{-up-modular} in $T$ when $k\geq k\br{T,c}$. Whenever clear from the context, we will use $k\br{T,c}$, $k\br{T}$, or $k$ to denote the lowest value such that $c$ is $k$-up-modular in $T$. To illustrate the notion of $k$-up-modularity, see Figure \ref{kUpModularityExample}.

The concept of $k$-up-modularity is a direct generalization of the notion of up-monotonicity of the cost function introduced in \cite{dereniowski2022CFApproxAlgForBSInTsWithMonoQTimes} (as monotonicity) and in \cite{dereniowski2024SInTsMonoQTs} (as up-monotonicity). Let $z=\argmax_{v\in V\br{T}}\brc{c\br{v}}$. A function $c$ is \textit{up-monotonic} in $T$ if for every $v,u\in V\br{T}$, whenever $v$ lies on the path between $z$ and $u$, we have $c\br{v}\geq c\br{u}$. 

It is easy to see that 1-up-modularity is equivalent to up-monotonicity. Observe that if $c$ is up-monotonic in $T$, then for every $t\in\mathbb{R}_{\geq 0}$, $T[V\br{T}-\brc{v\in V\br{T}\mid c\br{v}\leq t}]$ is connected and forms a single heavy module. Conversely, let $r=\argmax_{v\in V\br{T}}\brc{c\br{v}}$ and $u$ be any other vertex. If $c$ is 1-up-modular in $T$, then there is no vertex $v$ on the path between $r$ and $u$ such that $c\br{v}<c\br{u}$. Otherwise, for any $t\in \br{c\br{v},c\br{u}}$, $v$ does not belong to any heavy module, but $u$ and $r$ do. Since $v$ lies between them, $\spr{\mathcal{H}_{T,c}\br{t}}>1$, a contradiction.


\section{The parametrized $O\br{\log\log n}$-approx. solution}\label{parametrizedSolution}
\subsubsection{Cost levels}\label{costLevels}
The main idea of the algorithm is to partition vertices into intervals called \textit{cost levels} and process them in a top-down manner. At each level of the recursion, the algorithm schedules all necessary queries to vertices belonging to the given cost level. The rest of the decision tree is then built recursively. We consider the following intervals\footnote{We present the intervals in the ascending order in which a complete solution for each of them is obtained. However, since the procedure is recursive, the order in which the recursive calls are made is reverse.}:

\begin{enumerate}
    \item Firstly, an interval $\left( 0,{1}/{\log n}\right]$.
    \item Then, each subsequent interval $\mathcal{I}'=\left(a',b'\right]$ starts at the left endpoint of the previous interval $\mathcal{I}=\left(a,b\right]$, that is, $a'=b$, and ends with $b'=\min\brc{2b, 1}$. 
    
    This results in the following sequence of intervals, which partitions the interval $\left(0,1\right]$:
    
    $$\left( 0,{1}/{\log n}\right],\left({1}/{\log n},{2}/{\log n}\right], \left({2}/{\log n},{4}/{\log n}\right],\dots, \left({2^{\cl{\log\log n}-1}}/{\log n},1\right].$$
\end{enumerate}

We will ensure that when we call our procedure with parameters $\br{T, c, \left({2^{\cl{\log\log n}-1}}/{\log n},1\right]}$, the returned decision tree will be a valid decision tree for $T$.

We are now ready to introduce the notions of heavy and light vertices (and queries to them). We say that a vertex $v$ (or a query to it) is \textit{heavy} with respect to the interval $\mathcal{I}=\left(a,b\right]$ when $c\br{v}>a$. Otherwise, i.e., if $c\br{v}\leq a$, the vertex (and the query to it) is \textit{light} with respect to $\mathcal{I}$. Note that each heavy vertex belongs to some heavy module. Whenever clear from the context, we will omit the phrase "with respect to" and simply call the vertices and queries heavy and light.


\subsubsection{The main recursive procedure}\label{mainRecursiveProcedure}

We are ready to present the main recursive procedure. To avoid ambiguity, let $\mathcal{T}$ be the subtree of $T$ processed at some level of the recursion. Alongside $\mathcal{T}$ and a cost function $c$, the algorithm takes as input an interval $\left(a,b\right]$, such that for every $v\in V\br{\mathcal{T}}$, $c\br{v}\leq b$ and $2a\geq b$. 
The basic steps of the Algorithm \ref{createDecisionTree} are as follows: 
\begin{enumerate}
    \item If every vertex is heavy, return a decision tree built by calling the \FRankingBasedDT procedure for $\mathcal{T}$.
    \item Otherwise, find a set $\mathcal{Z}$, such that each connected component of $\mathcal{T}'\in \mathcal{T}-\mathcal{Z}$ contains at most one heavy module.
    \item Create an auxiliary tree $T_{\mathcal{Z}}$ using the vertices of $\mathcal{Z}$ and create a new decision tree $D_{\mathcal{Z}}$ for $T_{\mathcal{Z}}$, using the QPTAS from \cite{dereniowski2017ApproxSsForGeneralBSinWTs}.
    \item For each $\mathcal{T}'\in \mathcal{T}-\mathcal{Z}$, build a decision tree $D_H$, by calling the \FRankingBasedDT procedure for $\mathcal{T}'\angl{H}$. Then, hang $D_H$ below the last query to $v\in N_{\mathcal{T}}\br{\mathcal{T}'}$ in $D_{\mathcal{Z}}$.
    \item For each $L\in\mathcal{T}'-H$, build a decision tree recursively. Then, hang $D_L$ below the last query to a vertex $v \in N_{\mathcal{T'}}\br{L}$ in $D_{\mathcal{Z}}$.
    \item Return the resulting decision tree $D$.
\end{enumerate}

Before providing a detailed description and analysis of the above procedure, we first present some basic properties necessary for the subsequent considerations. In particular, we will make use of the following well-known lemma \cite{Cicalese2016DecTreesSimEval}:
\begin{lemma}\label{subtreeOptLemma}
    Let $T'$ be a subtree of $T$. Then, $\OPT\br{T'}\leq\OPT\br{T}$.
\end{lemma}
\begin{algorithm}
\caption{The main recursive procedure}\label{createDecisionTree}
\SetKwProg{Fn}{Procedure}{:}{}
\Fn{$\FCreateDecisionTree\br{\mathcal{T}, c, \left(a,b\right]}$}{
    \If{\textnormal{$b \leq 1/\log n$ \textbf{or} for every $ v\in V\br{\mathcal{T}}, c\br{v} > a$ }\tcp*{Every $v\in\mathcal{T}$ is heavy}}{
        \Return $\FRankingBasedDT\br{\mathcal{T}}$\label{basecaseDT} \tcp*{Apply Corollary \ref{vertexRankingCorollary}}
    }
    \Else{
        $\mathcal{X}\gets\emptyset$.

        \ForEach{$H\in\mathcal{H}_{\mathcal{T}, c}\br{a}$}{
            Pick arbitrary $v\in H$.
            
            $\mathcal{X}\gets\mathcal{X}\cup\brc{v}$.
        }

        $\mathcal{Z}\gets\mathcal{Y}\gets\mathcal{X}\cup\brc{v\in V\br{\mathcal{T}\angl{\mathcal{X}}} \mid \deg_{\mathcal{T}\angl{\mathcal{X}}}\br{v}\geq 3}$.

        \ForEach{\textnormal{$u,v\in \mathcal{Y}$ with $\mathcal{P}_{\mathcal{T}}\br{u,v}\neq\emptyset$ and $\mathcal{P}_{\mathcal{T}}\br{u,v}\cap \mathcal{Y}=\emptyset$}}{
            $\mathcal{Z}\gets \mathcal{Z}\cup\brc{\argmin_{z\in \mathcal{P}_{\mathcal{T}}\br{u, v}}\brc{c\br{z}}}$.  \tcp*{Lightest vertex on path}
        }

        $\mathcal{T}_{\mathcal{Z}} \gets \br{\mathcal{Z}, \brc{uv \mid \mathcal{P}_{\mathcal{T}}\br{u,v}\cap\mathcal{Z}=\emptyset}}$ \tcp*{Build auxiliary tree}

        $D \gets D_{\mathcal{Z}} \gets \FQPTAS\br{\mathcal{T}_{\mathcal{Z}}, c, \epsilon = 1}$\label{QPTAScall} \tcp*{Apply Theorem \ref{QPTAS}}

        \ForEach{$\mathcal{T}' \in \mathcal{T}-\mathcal{Z}$}{
            $H \gets$ the unique heavy module in $\mathcal{T}'$.

            $D_H \gets \FRankingBasedDT\br{\mathcal{T}'\angl{H}}$ \tcp*{Apply Corollary \ref{vertexRankingCorollary}}

            Hang $D_H$ in $D$ below the last query to $v \in N_{\mathcal{T}}\br{\mathcal{T}'}$ \tcp*{By Obs. \ref{neighborsPathObservation}}

            \ForEach{$L \in \mathcal{T}' - H$}{
                $D_L \gets \FCreateDecisionTree\br{L, c, \left(a/2,a\right]}$.\label{recursion}

                Hang $D_L$ in $D$ below the last query to $v \in N_{\mathcal{T}'}\br{L}$ \tcp*{By Obs. \ref{neighborsPathObservation}}
            }
        }

        \Return $D$.
    }
}
\end{algorithm}


For the rest of the analysis, fix $\mathcal{H}=\mathcal{H}_{\mathcal{T},c}\br{a}$ to be the set of heavy modules in $\mathcal{T}$. We have the following observations, which will be useful in the description and analysis of the algorithm:

\begin{observation}\label{heavymodulesetsize}
Let $\mathcal{H}$ be the set of heavy modules in $T$. Then, $\spr{\mathcal{H}}\leq k\br{T}$.
\begin{proof}
    Since $\mathcal{H}=\mathcal{H}_{\mathcal{T},c}\br{a}$, we have $\spr{\mathcal{H}}=k\br{\mathcal{T}, c, a}\leq \max_{t\in \mathbb{R}_{\geq 0}}k\br{T, c, t} = k\br{T, c}$.
\end{proof}
\end{observation}

\begin{observation}\label{subtreeKUpModularity}
    Let $T'$ be a subtree of $T$. Then, $k\br{T'}\leq k\br{T}$.
    \begin{proof}
        Fix any $t\in \mathbb{R}_{\geq 0}$ and let $H\in \mathcal{H}_{T, c}\br{t}$. We show that each such $H$ contributes at most 1 to $k\br{T', c, t}$. If $H\cap V\br{T'} = \emptyset$, then $H$ contributes 0. Otherwise, $H\cap V\br{T'}$ forms a connected subtree of $T'$, and thus contributes at most 1. The lemma follows.
    \end{proof}
\end{observation}

\begin{observation}\label{subtreePartialDt}
    Let $T'$ be a subtree of a tree $T$ and let $D'$ be a decision tree for $T'$. Then, $D'$ is a partial decision tree for $T$.
\end{observation}

\begin{observation}\label{neighborsPathObservation}
     Let $T'$ be a subtree of a tree $T$ and let $D$ be a partial decision tree for $T$ having no queries to vertices of $T'$, but containing at least one query to the vertices of $N_{T}\br{V\br{T'}}$. Let $Q$ denote the set of all such queries to vertices of $N_{T}\br{V\br{T'}}$ in $D$. Then, $D\angl{Q}$ forms a path in $D$.  
    \begin{proof}
        Let $q$ be any query in $D$. There are two cases:
        \begin{enumerate}
            \item $q\in V\br{T-V\br{T'}-N_{T}\br{V\br{T'}}}$. Then, for every $x\in N_{T}\br{V\br{T'}}$ being the target, $x$ belongs to the same connected component of $T-q$. Thus, no matter which vertex is the target, the answer is always the same. Therefore, $q$ has at most one child $u$ in $D$, such that $V\br{D_u}\cap Q \neq \emptyset$.
            \item $q\in N_{T}\br{V\br{T'}}$. After a query to $q$, the situation is as in the first case, except when $x=q$. Then, the response is $x$ itself, so no further queries are needed, and again $q$ has at most one child $u$ in $D$, such that $V\br{D_u}\cap Q \neq \emptyset$.
        \end{enumerate} 
        
        Since each $q\in Q$ has at most one child $u$ in $D$, with $D_u\cup Q\neq\emptyset$, $D\angl{Q}$ forms a path and the claim follows.
    \end{proof}
\end{observation}

\subsubsection{Base of the recursion}
We begin the description of our algorithm with the recursion base, which occurs whenever $b\leq{1}/{\log n}$ or for every $v\in V\br{\mathcal{T}}$, $c\br{v}>a$, i.e., every vertex is heavy. In such a situation, a solution is built by disregarding the costs of vertices and constructing a decision tree using the vertex ranking of $\mathcal{T}$. 

\begin{lemma}\label{baseOfRecursion}
    Let $D$ be a decision tree built, by calling \FRankingBasedDT$\br{\mathcal{T}}$ in line \ref{basecaseDT} of the \FCreateDecisionTree procedure. Then,  
$$
\COST_{D}\br{\mathcal{T}}\leq 2\cdot\OPT\br{T}.
$$
\begin{proof}
    There are two cases:
    \begin{enumerate}
        \item If $b\leq \frac{1}{\log n}$, then:
$$
\COST_{D}\br{\mathcal{T}}\leq\frac{\fl{\log n}+1}{\log n}\leq \frac{\log n+1}{\log n}\leq2\leq 2\cdot \OPT(\mathcal{T})\leq 2\cdot\OPT(T),
$$

where the first inequality is due to Corollary \ref{vertexRankingCorollary}, the fourth inequality follows from Observation \ref{basicBoundsOnCost}, and the last inequality is due to Observation \ref{subtreeOptLemma}.

        \item If for every $v\in V\br{\mathcal{T}}$, we have $c\br{v}> a$, then, define $c'\br{v}=a$ for all $v\in V\br{\mathcal{T}}$ (note that any value could be chosen here, since we treat each query as unitary). As $2c'\br{v} = 2a\geq b \geq c\br{v}$, we obtain $2\cdot \COST_D\br{\mathcal{T}, c'}\geq \COST_D\br{\mathcal{T}, c}$. Additionally, using the fact that $c'\br{v} \leq c\br{v}$, we have $\OPT\br{\mathcal{T}, c'}\leq \OPT\br{\mathcal{T}, c}$. Therefore:
$$
\COST_D\br{\mathcal{T}, c}\leq 2\cdot \COST_D\br{\mathcal{T}, c'}=2\cdot \OPT\br{\mathcal{T}, c'}\leq 2\cdot\OPT\br{\mathcal{T}, c}\leq 2\cdot\OPT\br{T, c},
$$

where the equality is due to Corollary \ref{vertexRankingCorollary} and the last inequality is due to Observation \ref{subtreeOptLemma}. The lemma follows.
    \end{enumerate}
\end{proof}
\end{lemma}


\begin{figure}[htp]
    \begin{minipage}[t]{0.45\textwidth}
    \centering
    \begin{tikzpicture}[every node/.style={draw, very thick}, every path/.style={very thick}]

    \node[circle, draw, minimum size = 0.75cm, fill=gray, drop shadow] (26) at (-3,-2) {};
    \node[circle, draw, minimum size=0.15cm, inner sep=0pt, fill=black] (27) at (-2.95,-1.95) {};
    \node[circle, draw, inner sep=0pt, fill=white] (41) at (-0.67,-2.433) {};
    \node[circle, draw, inner sep=0pt, fill=white] (42) at (-0.27,-2.38) {};
    \node[circle, draw, inner sep=0pt, fill=white] (43) at (-0.4,-1.57) {};
    
    \node[circle, draw, minimum size = 0.9cm, fill=gray, drop shadow] (28) at (-0.5,-2) {};
    \node[circle, draw, minimum size=0.15cm, inner sep=0pt, fill=black] (29) at (-0.5,-2) {};

    
    \node[circle, draw, minimum size=0.15cm, inner sep=0pt, fill=white] (30) at (1,-2.5) {};
    
    \node[circle, draw, minimum size = 0.75cm, fill=gray, drop shadow] (31) at (1.5,-3) {};
    \node[circle, draw, minimum size=0.15cm, inner sep=0pt, fill=gray!55] (32) at (1.4,-3) {};
    \node[circle, draw, inner sep=0pt, fill=white] (37) at (1.25,-3.3) {};
    \node[circle, draw, minimum size=0.15cm, inner sep=0pt, fill=black] (38) at (1.6,-2.79) {};
    \node[circle, draw, minimum size=0.15cm, inner sep=0pt, fill=white] (39) at (1.67,-3.025) {};
    
    \node[circle, draw, minimum size=0.15cm, inner sep=0pt, fill=white] (33) at (1.75,-3.5) {};
    
    \node[circle, draw, minimum size = 0.75cm, fill=gray, drop shadow] (34) at (1.9,-4) {};
    \node[circle, draw, minimum size=0.15cm, inner sep=0pt, fill=black] (35) at (2.05,-3.95) {};
    
    \node[circle, draw, minimum size = 1cm, fill=gray, drop shadow] (1) at (-3,2.75) {};
    \node[circle, draw, minimum size=0.15cm, inner sep=0pt, fill=black] (2) at (-3,3) {};

    
    
    \node[circle, draw, minimum size = 1.35cm, fill=gray, drop shadow] (3) at (-1,2.25) {};
    \node[circle, draw, minimum size=0.15cm, inner sep=0pt, fill=black] (4) at (-0.85,2.5) {};
    \node[circle, draw, inner sep=0pt, fill=white] (36) at (-0.3,2.25) {};
    \node[circle, draw, inner sep=0pt, fill=white] (40) at (-1.35,1.66) {};
    

    
    \node[circle, draw, minimum size=0.15cm, inner sep=0pt, fill=white] (5) at (-2.77,2) {};
    
    \node[circle, draw, minimum size=0.15cm, inner sep=0pt, fill=white] (6) at (-0.7,1.25) {};
    
    \node[circle, draw, minimum size = 0.75cm, fill=gray, drop shadow] (7) at (1.7,0.9) {};
    \node[circle, draw, minimum size=0.15cm, inner sep=0pt, fill=black] (8) at (1.75,0.95) {};
    
    \node[circle, draw, inner sep=0pt, fill=white] (9) at (-2.53,1) {};
    
    \node[circle, draw, minimum size=0.15cm, inner sep=0pt, fill=white] (10) at (1,0.65) {};
    
    \node[circle, draw, inner sep=0pt, fill=white] (11) at (0.5,0.47) {};
    
    \node[circle, draw, minimum size=0.15cm, inner sep=0pt, fill=gray!55] (12) at (-0.52,0.1) {};
    
    \node[circle, draw, minimum size=0.15cm, inner sep=0pt, fill=white] (13) at (-0.95,-0.1) {};

    
    \node[circle, draw, minimum size = 1cm, fill=gray, drop shadow] (14) at (-2,-0.5) {};
    \node[circle, draw, minimum size=0.15cm, inner sep=0pt, fill=black] (15) at (-2.2,-0.4) {};
    \node[circle, draw, minimum size=0.15cm, inner sep=0pt, fill=white] (16) at (-2,-0.45) {};
    \node[circle, draw, minimum size=0.15cm, inner sep=0pt, fill=gray!55] (17) at (-1.8,-0.5) {};
    
    \node[circle, draw, minimum size=0.15cm, inner sep=0pt, fill=white] (18) at (-1.35,-1.1) {};

    \node[circle, draw, minimum size=0.15cm, inner sep=0pt, fill=white] (19) at (-2.4,-1.25) {};
    
    \node[circle, draw, inner sep=0pt, fill=white] (20) at (-1.05,-1.45) {};
    
    \node[circle, draw, minimum size = 0.6cm, fill=gray, drop shadow] (21) at (1.5,-1.55) {};
    \node[circle, draw, minimum size=0.15cm, inner sep=0pt, fill=black] (22) at (1.5,-1.55) {};
    
    \node[circle, draw, minimum size=0.15cm, inner sep=0pt, fill=white] (23) at (1,-1.65) {};
    
    \node[circle, draw, minimum size=0.15cm, inner sep=0pt, fill=gray!55] (24) at (0.5,-1.775) {};
    
    \node[circle, draw, minimum size=0.15cm, inner sep=0pt, fill=white] (25) at (0.25,-1.85) {};

    
    
    \draw[] (2) to (5);
    \draw[] (4) to (6);
    \draw[] (5) to (9);
    \draw[] (6) to (12);
    \draw[] (8) to (10);
    \draw[] (9) to (15);
    \draw[] (10) to (11);
    \draw[] (11) to (12);
    \draw[] (12) to (13);
    \draw[] (13) to (17);
    \draw[] (15) to (16);
    \draw[] (16) to (17);
    \draw[] (17) to (18);
    \draw[] (17) to (19);
    \draw[] (18) to (20);
    \draw[] (19) to (27);
    \draw[] (20) to (29);
    \draw[] (22) to (23);
    \draw[] (23) to (24);
    \draw[] (24) to (25);
    \draw[] (24) to (30);
    \draw[] (25) to (29);
    \draw[] (30) to (32);
    \draw[] (32) to (33);
    \draw[] (32) to (39);
    \draw[] (33) to (35);
    \draw[] (38) to (39);
    \draw[very thick, fill=gray!20, drop shadow]
    (9).. controls +(-120:1.5cm) and +(-200:1.5cm).. (9);
    \draw[very thick, fill=gray!20, drop shadow]
    (11).. controls +(-30:3cm) and +(-110:3cm).. (11);
    \draw[very thick, fill=gray!20, drop shadow]
    (20).. controls +(-105:2cm) and +(-185:2cm).. (20);
    \draw[very thick, fill=gray!20, drop shadow]
    (36).. controls +(30:3.6cm) and +(-50:3.6cm).. (36);
    \draw[very thick, fill=gray!20, drop shadow]
    (37).. controls +(-80:2cm) and +(-160:2cm).. (37);
    \draw[very thick, fill=gray!20, drop shadow]
    (40).. controls +(-80:1.75cm) and +(-160:1.75cm).. (40);
    \draw[very thick, fill=gray!20, drop shadow]
    (41).. controls +(-70:3cm) and +(-150:3cm).. (41);
    \draw[very thick, fill=gray!20, drop shadow]
    (42).. controls +(-20:1.5cm) and +(-100:1.5cm).. (42);
    \draw[very thick, fill=gray!20, drop shadow]
    (43).. controls +(120:1cm) and +(30:1cm).. (43);
    
    \end{tikzpicture}
    \caption{Example tree $\mathcal{T}$. Dark grey circles represent heavy modules. Light grey regions represent light subtrees. Black vertices represent $\mathcal{X}$. Gray and black vertices represent $\mathcal{Y}$. White, gray and black vertices represent $\mathcal{Z}$. Lines represent paths of vertices between vertices of $\mathcal{Z}$.}\label{exampleTreeWithSetZ}
\end{minipage}
    \hspace{0.06\textwidth} % Add space between the two minipages
    \begin{minipage}[t]{0.45\textwidth}
    \centering
    \begin{tikzpicture}[every node/.style={draw, very thick, drop shadow}, every path/.style={very thick}]
    
    \node[circle, draw, minimum size=0.3cm, inner sep=0pt, fill=black] (2) at (-3,4) {};

    
    
    \node[circle, draw, minimum size=0.3cm, inner sep=0pt, fill=black] (4) at (-0.85,3.5) {};
    

    
    \node[circle, draw, minimum size=0.3cm, inner sep=0pt, fill=white] (5) at (-2.77,3) {};
    
    \node[circle, draw, minimum size=0.3cm, inner sep=0pt, fill=white] (6) at (-0.7,2.25) {};
    
    \node[circle, draw, minimum size=0.3cm, inner sep=0pt, fill=black] (8) at (2,2) {};
    
    
    \node[circle, draw, minimum size=0.3cm, inner sep=0pt, fill=white] (10) at (1,1.65) {};
    
    
    \node[circle, draw, minimum size=0.3cm, inner sep=0pt, fill=gray!55] (12) at (-0.52,1.1) {};
    
    \node[circle, draw, minimum size=0.3cm, inner sep=0pt, fill=white] (13) at (-0.95,0.9) {};

    
    \node[circle, draw, minimum size=0.3cm, inner sep=0pt, fill=black] (15) at (-2.5,0.6) {};
    \node[circle, draw, minimum size=0.3cm, inner sep=0pt, fill=white] (16) at (-2,0.55) {};
    \node[circle, draw, minimum size=0.3cm, inner sep=0pt, fill=gray!55] (17) at (-1.5,0.5) {};
    
    \node[circle, draw, minimum size=0.3cm, inner sep=0pt, fill=white] (18) at (-1.35,-0.1) {};

    \node[circle, draw, minimum size=0.3cm, inner sep=0pt, fill=white] (19) at (-2.4,-0.25) {};
    
    
    \node[circle, draw, minimum size=0.3cm, inner sep=0pt, fill=black] (22) at (1.5,-0.55) {};
    
    \node[circle, draw, minimum size=0.3cm, inner sep=0pt, fill=white] (23) at (1,-0.65) {};
    
    \node[circle, draw, minimum size=0.3cm, inner sep=0pt, fill=gray!55] (24) at (0.5,-0.775) {};
    
    \node[circle, draw, minimum size=0.3cm, inner sep=0pt, fill=white] (25) at (0,-0.85) {};

    \node[circle, draw, minimum size=0.3cm, inner sep=0pt, fill=black] (27) at (-2.95,-0.95) {};
    
    \node[circle, draw, minimum size=0.3cm, inner sep=0pt, fill=black] (29) at (-0.5,-1) {};

    
    \node[circle, draw, minimum size=0.3cm, inner sep=0pt, fill=white] (30) at (1,-1.5) {};
    
    \node[circle, draw, minimum size=0.3cm, inner sep=0pt, fill=gray!55] (32) at (1.4,-2) {};
    \node[circle, draw, minimum size=0.3cm, inner sep=0pt, fill=black] (38) at (1.9,-1.55) {};
    \node[circle, draw, minimum size=0.3cm, inner sep=0pt, fill=white] (39) at (1.8,-1.95) {};
    
    \node[circle, draw, minimum size=0.3cm, inner sep=0pt, fill=white] (33) at (2,-2.5) {};
    
    \node[circle, draw, minimum size=0.3cm, inner sep=0pt, fill=black] (35) at (2.55,-2.95) {};
    
    \draw[] (2) to (5);
    \draw[] (4) to (6);
    \draw[] (5) to (15);
    \draw[] (6) to (12);
    \draw[] (8) to (10);
    \draw[] (10) to (12);
    \draw[] (12) to (13);
    \draw[] (13) to (17);
    \draw[] (15) to (16);
    \draw[] (16) to (17);
    \draw[] (17) to (18);
    \draw[] (17) to (19);
    \draw[] (18) to (29);
    \draw[] (19) to (27);
    \draw[] (22) to (23);
    \draw[] (23) to (24);
    \draw[] (24) to (25);
    \draw[] (24) to (30);
    \draw[] (25) to (29);
    \draw[] (30) to (32);
    \draw[] (32) to (33);
    \draw[] (32) to (39);
    \draw[] (33) to (35);
    \draw[] (38) to (39);
    \end{tikzpicture}
    \caption{Auxiliary tree $\mathcal{T}_{\mathcal{Z}}$ built from vertices of set $\mathcal{Z}$.}\label{exampleAuxTreeTZ}
\end{minipage}
\begin{minipage}[t]{1\textwidth}
\centering
    \begin{tikzpicture}[scale=1.1]
    \draw[very thick, fill=black!80, drop shadow] 
    (4,-4) -- (4.9,-5.8) -- (3.1,-5.8) -- cycle; 
    
    \draw[very thick, fill=gray, drop shadow] (2.5,-6.5) -- (3.2,-7.9) -- (1.8,-7.9) -- cycle; 
    
    \draw[very thick, fill=gray, drop shadow] (3.8,-6.4) -- (4.3,-7.4) -- (3.3,-7.4) -- cycle; 
    
    \draw[very thick, fill=gray!20, drop shadow] (4.8,-6.3) -- (5.2,-7.1) -- (4.4,-7.1) -- cycle;
    
    \node at (5.75, -7) {$\dots$}; 
    
    \draw[very thick, fill=gray, drop shadow] (6.5,-6.5) -- (7.2,-8.1) -- (5.8,-8.1) -- cycle; 

    
    \draw[very thick] (3.1,-5.8) -- (2.5,-6.5);
    
    \draw[very thick] (3.6,-5.8) -- (3.8,-6.4); 
    
    \draw[very thick] (4.2,-5.8) -- (4.8,-6.3); 
    
    \draw[very thick] (4.9,-5.8) -- (6.5,-6.5); 

    \draw[very thick, fill=gray!20, drop shadow] (1.2,-8.5) -- (1.8,-9.7) -- (0.6,-9.7) -- cycle;

    \draw[very thick, fill=gray!20, drop shadow] (2.6,-8.4) -- (3.1,-9.4) -- (2.1,-9.4) -- cycle;
    
    \node at (3.25, -8.7) {$\dots$}; 

    \draw[very thick, fill=gray!20, drop shadow] (3.9,-8.3) -- (4.3,-9.1) -- (3.5,-9.1) -- cycle;
    
    \draw[very thick] (1.8,-7.9) -- (1.2,-8.5);
    \draw[very thick] (2.5,-7.9) -- (2.6,-8.4);
    \draw[very thick] (3.2,-7.9) -- (3.9,-8.3);

    \draw[very thick, fill=gray!20, drop shadow] (5,-8.8) -- (5.7,-10.2) -- (4.3,-10.2) -- cycle;

    \draw[very thick, fill=gray!20, drop shadow] (6.5,-8.7) -- (7.1,-9.9) -- (5.9,-9.9) -- cycle;
    
    \node at (7.2, -9) {$\dots$}; 

    \draw[very thick, fill=gray!20, drop shadow] (7.9,-8.6) -- (7.4,-9.6) -- (8.4,-9.6) -- cycle;
    
    \draw[very thick] (5.8,-8.1) -- (5,-8.8);
    \draw[very thick] (6.3,-8.1) -- (6.5,-8.7);
    \draw[very thick] (7.2,-8.1) -- (7.9,-8.6);

    \draw [decorate, 
       decoration = {calligraphic brace, amplitude = 10pt, mirror}, 
       line width = 1.5pt] 
      (0.5,-3.85) -- (0.5,-10.25);

    \node at (-0.6, -7.05) {$\COST_{D}\br{\mathcal{T}}$};
        
    \end{tikzpicture}
    \caption{The structure of the decision tree $D$, built by the Algorithm \ref{createDecisionTree}. The dark gray subtree represents the decision tree $D_{\mathcal{Z}}$, obtained by calling the QPTAS for  $\mathcal{T}_{\mathcal{Z}}$, $c$ and $\epsilon=1$. Gray subtrees represent decision trees $D_H$, each built for a unique heavy module $H\subseteq V\br{\mathcal{T}'}$ of every $\mathcal{T}'\in \mathcal{T}-\mathcal{Z}$, by calling the \textsc{RankingBasedDT} procedure for $\mathcal{T}'\angl{H}$. Light gray subtrees represent decision trees $D_L$, built for each $L\in \mathcal{T}'-H$, by recursively calling \textsc{CreateDecisionTree} with $L$, $c$ and $\left(a/2,a\right]$.}\label{structure_of_dt}
\end{minipage}
\end{figure}

\subsubsection{Construction of the Auxiliary Tree}\label{auxTreeConstruction}

To obtain the solution for the non-base case of our algorithm, we first construct the so-called \textit{auxiliary tree}. To do so, we begin by defining a set $\mathcal{X}\subseteq V\br{\mathcal{T}}$. For every heavy module $H\in\mathcal{H}$, we pick an arbitrary $v\in H$ and add it to $\mathcal{X}$. We also define a set 
$\mathcal{Y}=\mathcal{X}\cup\brc{v\in V\br{\mathcal{T}\angl{\mathcal{X}}} | \deg_{\mathcal{T}\angl{\mathcal{X}}}\br{v}\geq 3}$, by extending $\mathcal{X}$ to contain all vertices with degree at least $3$ in $T\angl{\mathcal{X}}$. Furthermore, we define a set $\mathcal{Z}\subseteq V\br{\mathcal{T}}$ consisting of the vertices in $\mathcal{Y}$ and, for every $u,v\in \mathcal{Y}$, such that $\mathcal{P}_{\mathcal{T}}\br{u, v}\neq\emptyset$ and $\mathcal{P}_{\mathcal{T}}\br{u, v}\cap \mathcal{Y}=\emptyset$, we add to $\mathcal{Z}$ the lightest vertex between them, i. e., 
$v_{u,v} = \argmin_{z\in \mathcal{P}_{\mathcal{T}}\br{u, v}}\brc{c\br{z}}$.
To see an example of construction of the sets $\mathcal{X}, \mathcal{Y}, \mathcal{Z}$, see Figure \ref{exampleTreeWithSetZ}. 

We then create the auxiliary tree 
$\mathcal{T}_{\mathcal{Z}}=\br{\mathcal{Z},\brc{uv | \mathcal{P}_{\mathcal{T}}\br{u, v}\cap \mathcal{Z}=\emptyset}}$ 
(for an example, see Figure \ref{exampleAuxTreeTZ}). Our algorithm starts by building a decision tree $D_{\mathcal{Z}}$ for $\mathcal{T}_{\mathcal{Z}}$, by taking $\epsilon=1$ and applying the QPTAS from Theorem \ref{QPTAS}. Observe that, since $D_{\mathcal{Z}}$ is a partial decision tree for $\mathcal{T}$ and corresponding vertices in $\mathcal{T}$ and $\mathcal{T}_{\mathcal{Z}}$ have the same costs, we have that:


\begin{observation}\label{CostDZinTObservation}
    $\COST_{D_{\mathcal{Z}}}\br{\mathcal{T}_{\mathcal{Z}}}=\COST_{D_{\mathcal{Z}}}\br{\mathcal{T}}$.
\end{observation}


Let $D = D_{\mathcal{Z}}$. For each connected component $\mathcal{T}'\in \mathcal{T}-\mathcal{Z}$, we build a new decision tree as follows: By the construction of $\mathcal{Z}$, all heavy vertices in $V\br{\mathcal{T}'}$ form a single heavy module $H\subseteq V\br{\mathcal{T}'}$. We create a new decision tree $D_H$ for $\mathcal{T}'\angl{H}$, by calling the \FRankingBasedDT procedure with argument $\mathcal{T}'\angl{H}$ and we hang $D_{H}$ in $D$ below the unique last query to a vertex in $N_\mathcal{T}\br{\mathcal{T}'}$ (which is possible due to Observation \ref{neighborsPathObservation}). As, by Observation \ref{subtreePartialDt}, $D_H$ is a partial decision tree for $\mathcal{T}'$, it follows that $D$ is also a partial decision tree for $\mathcal{T}$. 

Now notice that for each $L\in \mathcal{T}'-H$, there is no $v\in V\br{L}$, such that $c\br{v}>a$. This allows us to create a decision tree $D_L$ recursively, by calling the \FCreateDecisionTree procedure with arguments $L$, $c$ and $\left(a/2,a\right]$. Next, we hang $D_L$ in $D$ below the unique last query to a vertex in $N_{\mathcal{T}'}\br{L}$ (again, using Observation \ref{neighborsPathObservation}). Since after all such operations, every vertex $v\in V\br{\mathcal{T}}$ also belongs to $D$, we obtain a valid decision tree $D$ for $\mathcal{T}$. To see example structure of such solution, see Figure \ref{structure_of_dt}.


\subsubsection{Analysis of the algorithm}\begin{lemma}\label{auxTreeSizeLemma}
    Let $\mathcal{T}_{\mathcal{Z}}$ be the auxiliary tree. Then, $\spr{V\br{\mathcal{T}_{\mathcal{Z}}}}\leq 4k-3$.
    \begin{proof}
        First, we show that $\spr{\mathcal{Y}}\leq 2k-1$. We use induction on the elements of $\mathcal{H}$. We construct a family of sets $\mathcal{H}_1, \mathcal{H}_2, \dots, \mathcal{H}_{\spr{\mathcal{H}}}$, such that for every integer $1\leq h \leq \spr{\mathcal{H}}$, $\spr{\mathcal{H}_h}=h$ and $\mathcal{H}_{\spr{\mathcal{H}}} = \mathcal{H}$. For each $\mathcal{H}_h$, we also construct a corresponding set $\mathcal{Y}_h$, eventually ensuring that $\mathcal{Y}_{\spr{\mathcal{H}}}=\mathcal{Y}$.
        
        Let $\mathcal{H}_1=\emptyset$, $\mathcal{Y}_1=\emptyset$. Pick any heavy module $H\subseteq V\br{\mathcal{T}}$ and add it to $\mathcal{H}_1$. Add the unique vertex $v$, such that $v\in H\cap\mathcal{X}$ to $\mathcal{Y}_1$, so that $\spr{\mathcal{Y}_1}=1$. Assume by induction that for some $h\geq 1$, $\spr{\mathcal{Y}_h}\leq 2h-1$.  
        Two heavy modules $H_1,H_2\subseteq V\br{\mathcal{T}}$ will be called  \textit{neighbors} if for every $H_3\subseteq V\br{\mathcal{T}}$ with $H_3\neq H_1,H_2$, we have $\mathcal{P}_{\mathcal{T}}\br{H_1,H_2}\cap H_3 = \emptyset$. Pick $H\in\mathcal{H}$, such that $H\notin \mathcal{H}_h$ to be a heavy module that is a neighbor of some member of $\mathcal{H}_h$. We define $\mathcal{H}_{h+1}=\mathcal{H}_h\cup\brc{H}$. 
        Let $z$ be the unique vertex, such that $v\in H\cap\mathcal{X}$, and let $\mathcal{Y}_{h+1}=\mathcal{Y}_h \cup \brc{z}$. Define $\mathcal{T}_{h+1} = \mathcal{T}\angl{\brc{v\in \mathcal{Y}_{h+1} | \mathcal{P}_\mathcal{T}\br{v,z} \cap \mathcal{Y}_{h+1} = \emptyset}}$. Note that $\mathcal{T}_{h+1}$ is a spider (a tree with at most one vertex of degree above 2). Add to $\mathcal{Y}_{h+1}$ the unique vertex $v\in V\br{\mathcal{T}_{h+1}}$, such that $\deg_{\mathcal{T}_{h+1}}\br{v}\geq 3$, if it exists. Clearly, $\spr{\mathcal{Y}_{h+1}} \leq 2h+1$, completing the induction.  

        By construction, $\mathcal{H}_{\spr{\mathcal{H}}}=\mathcal{H}$ and $\mathcal{Y}_{\spr{\mathcal{H}}}=\mathcal{Y}$, so $\spr{\mathcal{Y}} \leq 2\cdot \spr{\mathcal{H}} - 1 \leq 2k-1$ where the last inequality is by Observation \ref{heavymodulesetsize}. As paths between vertices in $\mathcal{Y}$ form a tree when contracted, at most $2k-2$ additional vertices are added while constructing $\mathcal{Z}$ (at most one per path). The lemma follows.
    \end{proof}
\end{lemma}


\begin{lemma}\label{auxTreeCostLemma}
    Let $\mathcal{T}_{\mathcal{Z}}$ be the auxiliary tree. Then, $\OPT\br{\mathcal{T}_{\mathcal{Z}}}\leq \OPT\br{\mathcal{T}}$.
    \begin{proof}
        Let $D^*$ be the decision tree for $\mathcal{T}\angl{\mathcal{Z}}$. We build a new decision tree $D_{\mathcal{Z}}'$ for $\mathcal{T}_{\mathcal{Z}}$ by transforming $D^*$ as follows: 
        
        Let $u,v\in \mathcal{Y}$, such that $\mathcal{P}_{\mathcal{T}}\br{u, v}\neq\emptyset$ and $\mathcal{P}_{\mathcal{T}}\br{u, v}\cap \mathcal{Y}=\emptyset$. Let $q\in V\br{D^*}$ be the first query to a vertex among $\mathcal{P}_{\mathcal{T}}\br{u, v}$. Recall that we picked $v_{u,v}=\argmin_{z\in \mathcal{P}_{\mathcal{T}}\br{u, v}}\brc{c\br{z}}$, so $c\br{v_{u,v}}\leq c\br{q}$. 
        We replace $q$ in $D^*$ with the query to $v_{u,v}$ and delete all queries to vertices in $\mathcal{P}_{\mathcal{T}}\br{u, v}-v_{u,v}$. By construction, $D_{\mathcal{Z}}'$ is a valid decision tree for $\mathcal{T}_{\mathcal{Z}}$, and by choosing $v_{u,v}$ to minimize $c$, we did not increase the cost, so we have that: 
        $$
        \COST_{D_{\mathcal{Z}}'}\br{\mathcal{T}_{\mathcal{Z}}} \leq \OPT\br{\mathcal{T}\angl{\mathcal{Z}}}.
        $$
        
        Therefore, we have:
        $$
        \OPT\br{\mathcal{T}_{\mathcal{Z}}}\leq \COST_{D_{\mathcal{Z}}'}\br{\mathcal{T}_{\mathcal{Z}}}\leq \OPT\br{\mathcal{T}\angl{\mathcal{Z}}}\leq \OPT\br{\mathcal{T}},
        $$
        
        where the first inequality is due to the definition of optimality and the last inequality follows by Lemma \ref{subtreeOptLemma}.
    \end{proof}
\end{lemma}

\begin{lemma}\label{heavygroupcostlemma}
    Let $H$ be the unique heavy module of $\mathcal{T}'\in \mathcal{T}-\mathcal{Z}$. Then, the decision tree $D_H$ is of cost at most:
    $$\COST_{D_H}\br{\mathcal{T}'\angl{H}}\leq 2\cdot\OPT\br{\mathcal{T}}.$$
    \begin{proof}
        
    For every $v\in H$ let $c'\br{v}=a$. We have $2 c'\br{v}\geq b c'\br{v}/a =  b \geq c\br{v}$ so we get that $2\cdot\COST_{D_H}\br{\mathcal{T}'\angl{H}, c'}\geq \COST_{D_H}\br{\mathcal{T}'\angl{H}, c}$. Additionally, using the fact that $c'\br{v} \leq c\br{v}$ we have that $\OPT\br{\mathcal{T}'\angl{H}, c'}\leq \OPT\br{\mathcal{T}'\angl{H}, c}$. Hence:
    
    \begin{align*}
        \COST_{D_H}\br{\mathcal{T}'\angl{H}, c}&\leq 2\cdot\COST_{D_H}\br{\mathcal{T}'\angl{H}, c'}=2\cdot\OPT\br{\mathcal{T}'\angl{H}, c'} \\
        &\leq
        2\cdot\OPT\br{\mathcal{T}'\angl{H}, c}\leq 2\cdot\OPT\br{\mathcal{T}, c}
    \end{align*}
    
        
    where the equality is by the Corollary \ref{vertexRankingCorollary} and the last inequality is due to the fact that $\mathcal{T}'\angl{H}$ is a subtree of $\mathcal{T}'$, which is a subtree of $\mathcal{T}$ (Lemma \ref{subtreeOptLemma}).
    \end{proof}
\end{lemma}

\subsubsection{The main result}

Let $d$ be the remaining depth of the recursion call performed in Line~\ref{recursion} of the algorithm, i.e., the number of recursive steps from the current call to the base case (for the base case, this value is equal to $d=0$). We show that at each level of the recursion we pay $O\br{\OPT\br{T}}$, so the approximation ratio of the algorithm is bounded by $O\br{d}$:
\begin{lemma}\label{costofthesolution}
$\COST_D\br{\mathcal{T}}\leq\br{4 d+2}\cdot\OPT\br{T}$.
    \begin{proof}
        
    Let $Q_D\br{\mathcal{T},x}$ be the sequence of queries performed in order to find $x\in V\br{\mathcal{T}}$. By construction of the Algorithm \ref{createDecisionTree}, $Q_D\br{\mathcal{T},x}$ consists of at most three distinct subsequences of queries (see Figure \ref{structure_of_dt}):
    \begin{enumerate}
        \item Firstly, there is a sequence of queries belonging to $Q_{D_{\mathcal{Z}}}\br{\mathcal{T}_{\mathcal{Z}},x}$.
        \item If $x\notin \mathcal{Z}$, then, there is a sequence of queries belonging to $Q_{D_{H}}\br{\mathcal{T}'\angl{H},x}$ for a unique heavy group $H\subseteq V\br{\mathcal{T}'}$ of $\mathcal{T}'\in\mathcal{T}-\mathcal{Z}$, such that $x\in \mathcal{T}'$.
        \item At last, if $x\notin H$, there is a sequence of queries belonging to $Q_{D_{L}}\br{L,x}$ for $L\in \mathcal{T}'-H$, such that $x\in V\br{L}$. 
    \end{enumerate}  

    Note that it sometimes may happen that some of the above sequences are empty.
    
    We prove by induction that $\COST_{D}\br{\mathcal{T}}\leq \br{4d +2}\cdot \OPT\br{T}$. When $d=0$ (the base case), the induction hypothesis is true, due to the Lemma \ref{baseOfRecursion}. For $d>0$, assume by induction that the cost of the decision tree  built for each $L$, is at most $\COST_{D_L}\br{L}\leq \br{4\br{d-1} +2}\cdot \OPT\br{T}$. We have:
        \begin{align*}
        \COST_D\br{\mathcal{T}}
        &\leq
        \COST_{D_{\mathcal{Z}}}\br{\mathcal{T}}+\max_{\mathcal{T}'\in \mathcal{T}-\mathcal{Z}}\brc{\COST_{D_H}\br{\mathcal{T}'\angl{H}}+\max_{L\in \mathcal{T}'-H}\brc{\COST_{D_L}\br{L}}}
         \\ 
        &\leq
        \COST_{D_{\mathcal{Z}}}\br{\mathcal{T}_{\mathcal{Z}}}+\max_{\mathcal{T}'\in \mathcal{T}-\mathcal{Z}}\brc{2\cdot\OPT\br{\mathcal{T}}+\br{4\br{d-1} +2}\cdot \OPT\br{T}}
        \\ 
        &\leq 
        2\cdot\OPT\br{\mathcal{T}_{\mathcal{Z}}}+2\cdot\OPT\br{T}+ \br{4\br{d-1} +2}\cdot \OPT\br{T} 
        \\
        &
        \leq 
        2\cdot\OPT\br{\mathcal{T}}+4d\cdot \OPT\br{T} = \br{4d+2}\cdot \OPT\br{T}
        \end{align*}
        
        where the first inequality is due to the construction of the decision tree returned by the Algorithm \ref{createDecisionTree}, the second inequality is by Observation \ref{CostDZinTObservation}, Observation \ref{heavygroupcostlemma} and by the induction hypothesis, the third inequality is due to Theorem \ref{QPTAS} and using the fact that $\mathcal{T}$ is a subtree of $T$ (Lemma \ref{subtreeOptLemma}) and the last inequality is due to Lemma \ref{auxTreeCostLemma}.
        
    \end{proof}
\end{lemma}

We are now ready to prove our main theorem: 
\begin{theorem}
\label{parametrizedAlgorithm}
    There exists an $O\br{\log\log n}$-approximation algorithm for the Tree Search Problem running in $k^{O\br{\log k}}\cdot\text{poly}\br{n}$ time.
    \begin{proof}
        Let $D = \FCreateDecisionTree\br{T, \left({2^{\cl{\log\log n}-1}}/{\log n},1\right]}$. Since there are at most $\cl{\log\log n}+1$ intervals processed, the depth of the recursion is bounded by 
        $
        d \leq \cl{\log\log n} \leq \log\log n + 1
        $.
        Hence, using Lemma \ref{costofthesolution} we get that:
        $$
        \COST_{D}\br{T} \leq (4 \cdot \log\log n + 6) \cdot \OPT\br{T} = O\br{\log\log n \cdot \OPT\br{T}}.
        $$

        By Observation \ref{subtreeKUpModularity}, for every subtree $\mathcal{T}$ of $T$, processed at some level of the recursion, we have $k\br{\mathcal{T}} \leq k\br{T}$. Using Lemma \ref{auxTreeSizeLemma}, at each such level the call to the QPTAS from Theorem \ref{QPTAS} (line \ref{QPTAScall} of the \textsc{CreateDecisionTree}) runs in time bounded by: 
        $$
        k\br{\mathcal{T}}^{O\br{\log (4 \cdot k\br{\mathcal{T}})}} = k\br{T}^{O\br{\log k\br{T}}}.
        $$
        
        Since $d = O(\text{poly}(n))$ and all other computation can be performed in polynomial time, the overall running time is bounded by 
        $
        k^{O\br{\log k}} \cdot \text{poly}\br{n}
        $,
        as required.
    \end{proof}
\end{theorem}







