\newpage
\SetKwFunction{FRankingBasedDT}{RankingBasedDT}
\section{Trees, Worst Case, Uniform Costs}


% \subsubsection{Vertex ranking}\label{vertexRanking}
The \textit{vertex ranking} of $T$ is a labeling of vertices $l:V\to \brc{1,2,\dots,\cl{\log n}+1}$, which satisfies the following condition: for each pair of vertices $u,v\in V\br{T}$, whenever $l\br{u}=l\br{v}$, there exists $z\in \mathcal{P}_T\br{u,v}$ for which $l\br{z}>l\br{v}$. Such a labeling always exists and can be computed in linear time by means of dynamic programming \cite{Schaffer1989OptNodeRankOfTsInLinTime, OnakParys2006GenOfBSSInTsAndFLikePosets, Mozes_Onak2008FindOptTSStartInLinTime}. For a visual example, see Figure \ref{vertex_ranking_figure}

\begin{figure}[htbp]
    \centering
    \begin{minipage}{0.26\textwidth}
        \centering
        \tikz [tree layout, grow=-65,
               sibling distance=7mm, level distance=11mm,]
          \graph {
    ""[as=$a$] -- {
        ""[as=$b$] -- ""[as=$c$] -- {
            ""[as=$d$] -- ""[as=$e$],
            ""[as=$f$] -- { ""[as=$g$], ""[as=$h$], ""[as=$i$] }
        },
        ""[as=$j$] -- ""[as=$k$] -- { ""[as=$l$], ""[as=$m$] }
    }
};
    \caption[Sample tree with uniform costs]{}
    \label{vr_tree}
    \end{minipage}
    \begin{minipage}{0.26\textwidth}
        \centering
        \tikz [tree layout, grow=-65,
               sibling distance=7mm, level distance=11mm,]
          \graph {
    ""[as=$4$] -- {
        ""[as=$1$] -- ""[as=$3$] -- {
            ""[as=$2$] -- ""[as=$1$],
            ""[as=$2$] -- { ""[as=$1$], ""[as=$1$], ""[as=$1$] }
        },
        ""[as=$1$] -- ""[as=$2$] -- { ""[as=$1$], ""[as=$1$] }
    }
};
    \caption[Sample tree with uniform costs]{}
    \label{vr_ranking}
    \end{minipage}
    \begin{minipage}{0.45\textwidth}
        \centering
        \tikz [tree layout, grow=-90,
               sibling distance=8mm, level distance=15.5mm,]
            \graph {
              ""[as=$a$] -> {
                ""[as=$c$] -> {""[as=$b$], ""[as=$d$] -> {""[as=$e$]}, ""[as=$f$] -> {""[as=$g$], ""[as=$h$], ""[as=$i$]}}, ""[as=$k$]->{""[as=$j$], ""[as=$l$], ""[as=$m$]}
              }
            };
        \caption[Edge decision tree for a tree]{}
        \label{vr_dt}
    \end{minipage}
        \caption[Tree and decision trees for it]{Sample input tree $T$ (Figure \ref{vr_tree}), vertex ranking labeling $l$ of $T$ (Figure \ref{vr_ranking}) and a decision tree $D$ for $T$ built using $l$ (Figure \ref{vr_dt}).}
        \label{vertex_ranking_figure}
\end{figure}

Having a vertex ranking of $T$, one can easily obtain a decision tree for $T$ using the following procedure:
\begin{enumerate}
    \item Let $z\in V\br{T}$ be the unique vertex, such that for every $v\in V\br{T}$, $l\br{z}\geq l\br{v}$.
    \item Schedule a query to $z$ as the root of the decision tree $D$ for $T$.
    \item For each $T'\in T-z$, build a decision tree $D_{T'}$ recursively and hang it below the query to $z$ in $D$.
\end{enumerate} 

When the input tree has uniform costs and the ranking uses the minimal number of labels, the decision tree built in this way is optimal and never uses more than $\fl{\log n} + 1$ queries \cite{OnakParys2006GenOfBSSInTsAndFLikePosets}. Let \FRankingBasedDT be the name of the latter procedure. We have the following corollary:
  
\begin{corollary}\label{vertexRankingCorollary}
    There exists an $O\br{n}$ time procedure \FRankingBasedDT that finds the optimal decision tree for the Tree Search Problem when all costs are uniform. Moreover, the depth of such a decision tree, i.e., the worst-case number of queries, is at most $\fl{\log n} + 1$.
\end{corollary}

We assume that the input tree is rooted at an atribtarry vertex.
Below we show how to calculate the optimal vertex ranking of a given tree $T$. For any vertex $v\in V\br{T}$, and any coloring $l$ we define the \textit{visibility sequence} $S\br{v}$ as following: let $l\in \mathbb{N}$. If there exists a vertex $u\in V\br{T}$, such that for every $z\in\mathcal{P}_T\br{u,v}$, $l< l\br{z}$, then $l\in S\br{v}$. We also demand that $S\br{v}$ is sorted decreasingly. If for given label $l$, $l\in S\br{v}$, we say that such label is \texit{visible} from $v$. In order to find the coloring using the minimal amount of colors, we will make use of the standard lexicographic order on visibility sequences. Notice, that for any vertex $v\in V\br{T}$, we have that $\max_{u\in V\br{T}}\brc{l\br{u}}\in S\br{v}$. Therefore, a labeling with the lexycogrophically lowest $S\br{r\br{T}}$ is also an optimal. We focus ourselves on finding such labeling. We devise the following dynamic programming procedure which calculates a labeling of $T$ with a minimal visibility sequence of $r\br{T}$.

\SetKwFunction{FCalculateRanking}{CalculateRanking}
\begin{algorithm}
\caption{The \FCalculateRanking procedure.}\label{calculateRanking}
\SetKwProg{Fn}{Procedure}{:}{}
\Fn{$\FCalculateRanking\br{T}$}{
    \For{$1\leq j\leq \deg_{r\br{T}}^+$}{   
        $S\br{c_j}\gets \FCalculateRanking\br{T_{c_j}}$.
    }

    $m\gets$ maximal value belonging to at least two distinct $S\br{c_j}$.

    $S\br{v}\gets \bigcup_{j=1}^{\deg_{r\br{T}}^+} S\br{c_j}$.

    $l\br{r\br{T}}\gets \argmin_{m < k \leq \log n + 1, k \notin S\br{v}}\brc{k}$.

    Append $l\br{v}$ to $S\br{v}$.

    Remove any $l < l\br{v}$ from $S\br{v}$.

    \Return $S\br{v}$.
}
\end{algorithm}

The following theorem is by \cite{OnakParys2006GenOfBSSInTsAndFLikePosets}:
\begin{theorem}
    Let $T$ be a tree. The Algorithm \ref{calculateRanking} can be implemented in $O\br{n}$ time and returns a correct ranking labeling such that $S\br{r\br{T}}$ islexycogrophically optimal.
\end{theorem}

