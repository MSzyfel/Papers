\section{Graph theory}
A \textit{\gls{graph}} is a pair $G=\br{V\br{G}, E\br{G}}$ where $V\br{G}$ is the set of \textit{vertices} and $E\br{E}$ is the set of \textit{edges} which are unordered pairs of vertices. We denote $n\br{G}=\spr{V\br{G}}$ and $m\br{G}=\spr{E\br{G}}$. For $u,v \in V\br{G}$ by $uv$ we denote the edge which connects them. A \textit{subgraph} of a graph $G$ is another graph $G'$ formed from a subset of the vertices and edges of $G$. For any $V'\subseteq V\br{G}$ by $G[V']$ we denote the \textit{subgraph induced} by $V'$ in $G$ (i. e. for every $u,v\in V'$ if $uv\in E\br{G}$, then also $uv\in E\br{G'}$).  Additionally, by $G-V'$ we denote the set of connected components occurring after deleting all vertices in $V'$ from $G$. The set of \textit{neighbors} of $v\in V\br{G}$ will be denoted as $N_G\br{v} = \brc{u\in V\br{G}|uv\in E\br{G}}$ and the set of neighbors of subgraph $G'$ of $G$ as $N_G\br{G'} = \bigcup_{v\in V\br{G'}}N_G\br{v}-V\br{G'}$. By $\deg_{G}\br{v}=\spr{N_G\br{v}}$ we will denote the \textit{degree} of $v$ in $G$. By $\Delta\br{G} = \max_{v\in V\br{G}}\brc{\spr{\deg\br{v}}}$ we denote the degree of $G$.

A \textit{cycle} is a non-empty sequence of vertices in which for every two consecutive vertices $u,v$: $uv \in E\br{G}$ and only the first and last vertices are equal. A \textit{tree} $T$ is a connected graph that contains no cycle. A \textit{forest} is a (not necessarily connected) graph that contains no cycle. A \textit{path} $P$ is a tree such that $\Delta\br{P} = 2$. Let $v\in V\br{T}$. The \textit{outdegree} of $v$ in $T$ will be denoted as $\deg_T^+\br{v}=\spr{\mathcal{C}_T\br{v}}$.
By $P_{T}\br{u, v}=T\angl{\brc{u,v}}-\brc{u,v}$ we denote a path of vertices between $u$ and $v$ in $T$ (excluding $u$ and $v$). Analogously, for $V_1,V_2\in V\br{T}$ we define $P_{T}\br{V_1, V_2}=T\angl{V_1\cup V_2}-\br{V_1\cup V_2}$.
For any we denote the minimal connected subtree of $T$ containing all vertices from $V'$ by $T\angl{V'}$.

A partial ordering $\preceq$ is a two-argument relationship which is: reflective ($a\preceq a$), antisymmetric (if $a\preceq b$ and $b \preceq a$ then $a=b$) and transitive (if $a\preceq b$ and $b \preceq c$ then $a\preceq c$).
A poset is a pair $\mathcal{P}=\br{X, \preceq}$ where $X$ is the set of elements and $\preceq$ is a partial ordering of elements of in $X$. When clear from the context the set $X$ itself is also sometimes called a poset. 
